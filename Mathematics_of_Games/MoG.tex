\documentclass[a4paper, 12pt]{article}

\usepackage{fullpage}
\usepackage[utf8]{inputenc}
\usepackage[english]{babel}
\usepackage{amsmath,amssymb}
\usepackage[explicit]{titlesec}
\usepackage{ulem}
\usepackage[onehalfspacing]{setspace}
\usepackage{amsthm}

\theoremstyle{plain}
\newtheorem{theorem}{Theorem}[subsection] % reset theorem numbering for each chapter

\theoremstyle{definition}
\newtheorem{definition}[theorem]{Definition} % definition numbers are dependent on theorem numbers
\theoremstyle{lemma}
\newtheorem{lemma}[theorem]{Lemma}

\theoremstyle{remark}
\newtheorem{remark}[theorem]{Remark}

\theoremstyle{corollary}
\newtheorem{corollary}[theorem]{Corollary}

\theoremstyle{example}
\newtheorem{example}[theorem]{Example}

\titleformat{\subsection}
{\small}{\thesubsection}{1em}{\uline{#1}}
\begin{document}
	\begin{titlepage} 
		\title{Gae Summary}
		\clearpage\maketitle
		\thispagestyle{empty}
	\end{titlepage}
	\tableofcontents
	\newpage
	\section{Introduction}
	We will introduce four different models of game theory: \begin{enumerate}
		\item Static Model of Complete Information: This model only uses one game round where every player has access to all available information (e.g. coin matching).
		\item Dynamic Model of Complete Information: This is essentially the same as the first model but using many different rounds
		\item Bayesian Incomplete Information: Similar to the first model but a player may not have complete information about another players strategies (e.g. auctions). That means the game uses hidden information.
		\item perfect Bayesian Incomplete Information
	\end{enumerate}
	\section{Strategic games of Complete Information}
	\begin{definition}
		A game in strategic form has three components
		\begin{itemize}
			\item the set of (rational) players $P = \{1,2,...,n\}$
			\item a pure strategy space $S_i$ for player $i$. $S_i$ is the set of all possible decisions of $i$. All decisions are taken simultaneously and all information is common knowledge.
			\item a utility function $u_i$ for each player $p_i$ that returns a gain (or loss) for player $i$ for a given strategy vector $s = (s_1,s_2,...,s_n)$ which is also called a pure-strategy payoff.
		\end{itemize}
	\end{definition}
	\begin{example}
		Let $A = [0,1]$  and $P = \{0,1\}$. Each player picks a $x_i \in A$ and wants to maximize the measure of $S(x_i) = \{x \in A: \, \left|x-x_i\right| < \left|x-x_{1-i}\right|\}$. That means $u_i(x_i) = \lambda(S(x_i))$. The Nash equilibrium in this example is $x_0=x_1=\frac{1}{2}$. For three players there is no Nash equilibrium but for any other number there is.
	\end{example}
	\begin{definition}
		Let $i \in P$. We often use the simplified notation $s_{-i}$ as the pure-strategy form of all players except $i$. That is \[s_{-i} = (s_1,..,s_{i-1},s_{i+1},...,s_n)\]
	\end{definition}
	\begin{definition}
		A pure-strategy Nash Equilibrium $s^* = (s_1^*,s_2^*,...,s_n^*)$ is a strategy profile such that $\forall i \in P$ it holds \[u_i(s_i^*,s_{-i}^*) \geq u_i(s_i,s_{-i}^*) \; \forall s_i \in S_i\]
	\end{definition}
	\begin{definition}
		A pure strategy $s_i$ is strictly dominated for player $i$ if there exists \[\sigma_i' \in \Sigma_i\] such that \[u_i(\sigma_i',s_{-i}) > u_i(s_i,s_{-1})\] for all $s_{-i} \in S_{-i}$. $s_i$ is weakly dominated if there is a $\sigma_i'$ such that the inequality is weak but strict for at least one $s_i$.
	\end{definition}
	\begin{definition}[mixed strategy]
		A mixed strategy $\sigma_i$ is a probability distribution over pure strategies. The space of mixed strategies of player $i$ will be denoted by $\Sigma_i$ where $\sigma_i(s_i)$ is the probability that $\sigma_i$ chooses $s_i$. The space of mixed strategy profiles is $\Sigma = \times_i \Sigma_i$. The support of a mixed strategy is the set of pure strategies that are assigned a positive probability. The payoff of player $i$ under the profile $\sigma$ is \[u_i(s) = \sum_{s \in S} \left(\prod_{j=1}^{\left|P\right|} \sigma_j(s_j)\right)u_i(s)\]
	\end{definition}
	\begin{definition}[iterative elimination of strictly dominated strategies]
		Set $S_i^0 = si$ and $\Sigma_i^0 = \Sigma_i$. Now define $S_i^n$ recursively by \[S_i^n = \{s_i \in S_i^{n-1}: \text{there is no } \sigma_i \in \Sigma_i^{n-1} \text{ such that } u_i(\sigma_i,s_{-i}) > u_i(s_i,s_{-1})\}\] and define \[\Sigma_i^n = \{s_i \in \Sigma_i : \sigma_i(s_i)> 0 \text{ only if } s_i\in S_i^n\}\]
	\end{definition}
	\subsection{Monopoly Market: 1-Player Cournot}
	Assume there is only one player (e.g. a company) that produces one product where production for each item costs $c \in \mathbb{R}$. The company can choose the quantity $q$ to produce. Furthermore there is an $a \in \mathbb{R}$ s.t. for each item sold, the player gains $a-q$. This should model the principle of supply and demand. The question is now, how the profit can be maximised.\\
	We calculate the payoff function w.r.t. to $q$ \[u(q) = (a-c-q)q = q(a-c) - q^2\] We optimize via the first derivative \[0 \overset{!}{=} u'(q) = a-c-2q\] Hence $u$ is maximized at $q = \frac{a-c}{2}$.
	\subsection{Duopoly: 2-Player Cournot}
	The setting is the same as in the previous section, and the constant $c$ is the same for both companies. Both choose $q_i$ from $\mathbb{N}$. They can sell each item for a prize of $a-q_1-q_2$.\\
	Let $(q_1^*,q_2^*)$ be a strategy profile. We fix $q_2^*$ and compute the optimal value for $q_1^*$ (called the best response). \[\max_{q_1} u_1(q_1,q_2^*)\] where \[u_1(q_1,q_2^*) = q_1(a-q_2^*-c)-q_1^2\] and again using the first derivative we find \[q_1^* = \frac{a-q_2^*-c}{2}\] By symmetry one can find the same value for $q_2^*$ and by a substitution \[q_1^* = \frac{a-c}{3}\] By symmetry this must be the unique pure-strategy Nash equilibrium.
	\section{Dynamic Model of Complete Information}
	Let $A^0 = (a_1^0,...,a_n^0)$ be a stage-0 strategy profile. At the beginning of stage 1, know history $h^1$ which is simply $a^0$. The possible strategies for player $i$ are now potentially dependent on $h^1$, which is why we denote the set by $A_i(h^1)$. This is generalised for all stages $k$.\\
	A pure strategy is thus a plan on how to play in each stage $k$ for a possible history $h^k$. Let $H^k$ be the set of all possible stage-$k$ histories and let \[A_i(H^k) = \bigcup_{h^k \in H^k} A_i(h^k)\]
	The a pure strategy for player $i$ is a sequence of maps $\{s^k\}_{k=0}^K$ where each $s_i^k$ maps $H^k$ to their set of possible actions. \[s_i^k(h^k) \in A_i(h^k)\]
	The sequence of strategies is thus $a^0 = s^0(\varnothing)$, $a^1 = s^1(a^0)$, $a^2 = s^2(a^0,a^1)$, ....\\
	This is called the path of the strategy profile. The payoff function can be defined via \[u_i: H^{k+1} \to \mathbb{R}\] is usually some weighted average of the single-stage payoffs. The weight of stage $k$ is $0 \leq \delta^k \leq 1$ which implies the definition of the payoff $u_i$ \[u_i = \sum_{t=0}^{T-1} \delta_i^t u_i^t(a^t)\] where the payoff of stage $t$ is $u_i^t(a^t)$. A pure strategy NE is then a strategy profile $s$ such that no player $i$ can do better with a different strategy, i.e. \[\forall i \in P, u_i(s_i^*,s_{-i}^*) \geq u_i(s_i,s_{-i})\] for all possible $s_i$.
	\subsection{Subgame-perfect Nash Equilibrium}
	If we only look at the game starting from stage $k$ we denote this subgame by $G(h^k)$. Here, all players have common knowledge of the history $h^k$ and the payoff is $u_i(h^k,s^k,...,s^{T-1})$. Any strategy profile of the entire game induces a strategy profile on the subgame; for player $i$ $s_i|_{h^k}$ is the restriction of $s_i$ to the histories consistent with $h^k$. In this scenario, a strategy profile $s$ of a multi-stage game is a subgame-perfect equilibrium of for every $h^k$, the restriction $s|_{h^k}$ to $G(h^k)$ is a Nash equilibrium of $G(h^k)$.
	\subsection{Stackelberg Duopoly Market Game}
	Two companies need to choose a quantity $q_i \in [0,\infty)$. In stage 0, company 1 chooses $q_1$ while company 2 observes and stays quiet. In stage 1, company 2 chooses an answer $q_2$. As in the Cournot games we assume that both have the same cost $s$ and that there is a constant $a$ such that they can sell for $p(q) = a-q$ where $q = q_1+q_2$. It is easy to see that \[q_2 = \frac{a-c-q_1}{2}\] is the best reaction.\\
	However company 1 knows the best response to their first pick and can therefore maximize \[u_1(q_1,q_2) = u_1(q_1,\frac{a-c-q_1}{2})\] and chooses \[q_1 = \frac{a-c}{2}\]
	Thus the pure-strategy Nash Equilibrium of the game in stage 0 is $q_1 = \frac{a-c}{2}$ and in stage 1 with $h^1 = (q_1)$ it is $q_2 = \frac{a-c}{4}$.
	\subsection{Repeated Finite Prisoner's Dilemma}
	Since the payoff in each stage equals the sum of this stage and the payoff of the subgame before, we can notice that defecting stays the unique pure-strategy Nash Equilibrium. This is because the first stage is simply the one-round Prisoner's Dilemma. In the second stage, the payoff table changes to the usual table plus the payoff of the previous stage. Since this was zero (which is the Nash Equilibrium) the table doesn't change. This backwards induction can be continued and finally shows that defecting stays the Nash Equilibrium until the last round.\\
	There are however other Nash Equilibria if an infinite number of rounds are played.
	\subsection{Rubinstein-Stahl Bargaining Game}
	Two players have to agree on how to share a pie of size 1. In even periods, player 1 proposes a sharing of $(x,1-x)$ and player 2 accepts or rejects. If player 2 accepts at anytime, the game ends. If player 2 rejects, they can propose their own sharing in every odd period. If player 1 accepts any proposal, the game ends. Notice that periods and stages are not the same. A period consists of two stages. Thus if $(x,1-x)$ is accepted in period $t$ then the payoffs are $\delta_1^tx$ for player 1 and $\delta_2^t(1-x)$ for player 2.\\
	There is a unique subgame-perfect Nash Equilibrium. Player $i$ always demands a share of $\frac{(1-\delta_j)}{1-\delta_i\delta_j}$. They accept any share greater or equal to $\frac{\delta_i(1-\delta_j)}{1-\delta_i\delta_j}$ and refuses anything else. Note that $\frac{(1-\delta_j)}{1-\delta_i\delta_j}$ is the highest share for player $i$ that is accepted by player $j$. They cannot improve their payoff by proposing a smaller share because that will be accepted. Proposing a higher share for themselves would also decrease payoff, since it would be rejected and the payoff for player $i$ in the next round is \[\delta_i(1-\frac{1-\delta_i}{1-\delta_i\delta_j}) = \delta_i^2\frac{1-\delta_i}{1-\delta_i\delta_j} < \frac{1-\delta_i}{1-\delta_i\delta_j}\]
	\begin{definition}
		The continuation payoffs of a strategy profile models the possible payoff a player can achieve if they decide to keep playing (in this case that means refusing the other player's offer).
	\end{definition}
	Using this knowledge, we can prove the uniqueness of the Nash Equilibrium. Define $\underline{v}_i$ and $\overline{v}_i$ to be the lowest and highest continuation payoffs of player $i$ in any perfect equilibrium of any subgame that begins with player $i$ making an offer. Analogously, define $\underline{w}_j$ and $\overline{w}_j$ to be the highest and lowest values in subgames starting with $j$. When player 1 makes an offer, player 2 will accept any $x$ such that player 2's share $(1-x)$ exceeds $\delta_2\overline{v}_2$. Player 2 cannot expect more than $\overline{v}_2$ in the continuation game following their refusal. Hence \[\underline{v}_1 \geq 1-\delta_2\overline{v}_2\] By symmetry, player 1 accepts all shares above $\delta1\overline{v}_1$ and \[\underline{v}_2 \geq 1-\delta_1 \overline{v}_1\] Since player 2 will never offer player 1 a share greater that $\delta\overline{v}_1$, player 1's continuation payoff when player 2 makes an offer, $\overline{w}_1$ is at most $\delta_1\overline{v}_1$. Since player 2 can obtain at least $\underline{v}_2$ in the continuation game by rejecting player 1's offer, they will reject any $x$ such that $1-x < \delta_2\underline{v}_2$. Player 1's highest equilibrium payoff when making an offer $\overline{v}_1$ satisfies \[\overline{v}_1 \leq \max\{1-\delta_2\underline{v}_2, \delta_1,\overline{w}_1\} \leq \max\{1-\delta_2\underline{v}_2,\delta_1^2\overline{v}_1\}\] This last maximum is actually equal to $1-\delta_2\underline{v}_2$. Combining everything we have shown and doing some number crunching we get $\underline{v}_1 = \overline{v}_1$. Similarly one proves that $\underline{v}_2 = \overline{v}_2$ and $\underline{w}_i = \overline{w}_i$.
	\subsection{Infinite Cournot Duopoly}
	Consider the infinitely repeated Cournot-Duopoly game with discount factor $\delta$ and marginal cost $c$ for both companies. Assume that both players play the following strategy profile which delivers at each stage the social optimum.
	"Produce half the monopoly quantity $\frac{q_m}{2} = \frac{a-c}{4}$ in the first period. In the $t$-th period, produce $\frac{q_m}{2}$ if both players have produced $\frac{q_m}{2}$ in each of the $t-1$ previous periods. Otherwise produce the Cournot quantity $q_C = \frac{a-c}{3}$."\\
	We now aim to calculate the values for $\delta$ where the above strategy is a Nash Equilibrium. If both play $\frac{q_m}{2}$ then both have a payoff of $\frac{(a-c)^2}{8}$. If both play $q_C$, each player gets $\frac{(a-c)^2}{9}$. If player $i$ is going to produce $\frac{q_m}{2}$ in this stage, then the quantity that maximizes $j$'s profit in this stage is the solution to \[\max_{q_i} \left(a-q_j-\frac{q_m}{2}-c\right)q_j\] no matter the stage. This means, that this is valid for any subgame of the infinite multi-stage game. The solution is $q_j = \frac{3(a-c)}{8}$ with associated profit $\frac{9(a-c)^2}{64}$. Thus it is a NE for both companies to play the given strategy exactly if \[\frac{1}{1-\delta} (a-c)^2/8 \geq 9(a-c)^2/64 + \frac{\delta}{1-\delta}(a-c)^2/9 \Leftrightarrow \delta \geq \frac{9}{17}\]
	\subsection{Simple Time Stopping Game}
	In such a game, every player has the action $stop$ which can have positive or negative influence on their payoff. Once a player stops, they can no longer change their strategy. We denote the payoff of the first player to stop by $L(t)$. $F(t)$ is the payoff of the following player if there is one and $B(t)$ is the payoff for both players if they stop simultaneously in a 2-player game. Consider the following example:
	\begin{definition}[2-player Cat-Dog-Fight Stopping game]
		A cat and a dog are fighting for a prize whose current value at any time is $v>1$. Fighting costs 1 unit per period. If one animal stops fighting at period $t$ the other player wins the prize alone without incurring any cost in that period. The second stopping time is irrelevant. If both stop simultaneously no-one wins the prize. If we consider a per-period discount factor $\delta$ the symmetric payoff functions are \[L(t) = B(t) = -(1+\delta + ... + \delta^{t-1}) = -\frac{1-\delta^t}{1-\delta}\] for whoever stops first and \[F(t) = L(t)+\delta^tv\]
	\end{definition}
	We aim to find a subgame-perfect Nash equilibrium. Regardless of the period, we assume both players stop with probability $p$. For this symmetric mixed strategy to be an equilibrium, it is necessary, that \[L(t) = p\cdot F(t) + (1-p)L(t+1\\)\] Equating these gives $p^* = \frac{1}{1+v}$ which ranges from 0 to 1.
	\begin{theorem}[One-Stage-Deviation Condition for Finite Games]
		Consider a finite multi-stage game with observed actions. Then a strategy profile $s$ is a subgame perfect Nash Equilibrium iff it satisfies the one-stage-deviation condition that no player $i$ can gain by deviating from $s$ in a single stage and conforming to $s$ thereafter.\\
		More precisely, a profile $s$ is a subgame-perfect NE iff there is no player $i$ and no strategy $\hat{s}_i$ that agrees with $s_i$ except at a single stage $t$ and $h^t$ and such that $\hat{s}_i$ is a better response to $s_{-i}$ than $s_i$ if history $h^t$ is reached.
	\end{theorem}
	\begin{definition}[Continuous at Infinity Games]
		A game is continuous at infinity if for each player $i$ the utility function $u_i$ satisfies \[\lim_{t \to \infty} \sup_{h,\tilde{h} \,s.t.\, h^t = \tilde{h}^t} \left|u_i(h^t) - u_i(\tilde{h}^t)\right| = 0\]
	\end{definition}
	\section{Static Model of Incomplete Information}
	\subsection{Bayesian Games}
	In this kind of games, a player $i$ has a type $\theta_i$ drawn from an objective distribution $p$. The space of all possible $\theta_i$ is called $\Theta_i$. $p(\theta_{-i}|\theta_i)$ is the conditional probability of player $i$ about their opponents' types. The marginal distribution $p_i(\theta_i)$ is strictly positive. The space of pure strategies $s_i$ of player $i$ is denoted by $S_i$ and the mixed strategies $\sigma_i$ by $\Sigma_i$. Finally the payoff function is $u_i(s_1,...,s_n,\theta_1,...,\theta_n)$. Strategy spaces, payoff functions, type spaces and prior distributions are all common knowledge and al private information is included in the description of the player's type. The choice of a player is dependent on their type. $\sigma_i(\theta_i)$ is the strategy $i$ chooses when their type is $\theta_i$. Hence, if $i$ knew the strategies of the other players as a function of their type, they could use their beliefs $p(\theta_{-i}|\theta_1)$ to compute the expected utility and find their optimal response.
	\begin{definition}
		A pure-strategy Bayesian Nash Equilibrium in a Bayesian game (i.e. a game of incomplete information) with types $\theta_i$ for each player $i$, prior distribution $p$ and pure-strategy spaces $S_i$ is a NE of the expanded game in which each player $i$'s space of pure strategies is the set of $S_i^{\Theta_i}$ of maps from $\theta_i$ to $S_i$.\\
		Given a strategy profile $s(\cdot)$ and $s_i'(\cdot)\in S_i^{\Theta_i}$ let \[(S_i'(\theta_i),s_{-i}(\theta_{-i}))\] denote the value of the strategy profile $(s_i'(\cdot),s_{-i}(\cdot))$ evaluated at $\theta = (\theta_i, \theta_{-i})$. The $s(\cdot)$ is a pure-strategy Bayesian NE if 
	\end{definition}
	\subsection{perturbed matching coins game}
	The payoffs of both players is changed in one component.\\$u_1(H,H) = 1+\theta_1$ and $u_2(T,H) = 1+\theta_2$. The $\theta_i$ are from an interval $[-\theta,\theta]$. The symmetric pure strategies $(s_i(\theta_i \geq 0) = H, s_i(\theta_i < 0) = T)$ form a Bayesian Nash Equilibrium. From the point of view of each player, the other chooses either $H$ or $T$ with probability $\frac{1}{2}$. Thus the player should choose $H$ if and only if $\frac{1}{2}(1+\theta_i) + \frac{1}{2}(-1) \geq 0 \Leftrightarrow \theta_i \geq 0$. When $\theta$ converges to $0$ the BNE of the incomplete information game converges to the unique NE $(\frac{1}{2},\frac{1}{2})$ of the complete information game.
	\subsection{Perturbed battle of the sexes}
	Again, we change the payoff slightly to be $u_1(M,M) = 2+\theta_1$ and $u_2(F,F) = 2+\theta_2$. The $\theta_i$ are from an interval $[0,\theta]$. Player 1 plays $M$ if $\theta_1$ exceeds a critical value $c$ and $F$ otherwise. Player 2 plays $F$ if $\theta_2$ exceeds a critical value $p$ and $M$ otherwise. In such a strategy profile, player 1 plays $M$ with probability $\frac{\theta-c}{\theta}$ and player 2 plays $F$ with probability $\frac{\theta-p}{\theta}$. It is easy to see that player 1's expected payoff given player 2's strategy is \[\frac{p}{\theta}(2+\theta_1) + (1-\frac{p}{\theta})\cdot 0 = \frac{p}{\theta}(2+\theta_1)\] and \[\frac{p}{\theta}\cdot 0 + (1-\frac{p}{\theta})\cdot 1 = 1-\frac{p}{\theta}\] Therefore, playing $M$ is optimal if and only if $\theta_1 \geq \frac{\theta}{p}-3$. In the same way we can see that player 2 plays $F$ if and only if $\theta_2 \geq \frac{\theta}{c}-3$. Thus we get $p=c$ and $p^2 + 3p -\theta = 0$. From this we can get that the probability that player 1 plays $M$ and the probability that player 2 plays $F$ are both equal to \[1-\frac{-3 + \sqrt{9+4\theta}}{2\theta}\] which goes to $\frac{2}{3}$ as $\theta$ goes to $0$.
	\subsection{Public Good Game}
	In this game two players have the pure strategies CONTRIBUTE or DON'T (abbreviated by C and D). The payoffs are \[u_1(C,C) = 1-c_1, \; u_1(C,D) = 1-c_1, \; u_1(D,C) = 1, \; u_1(D,D) = 0\]
	and \[u_1(C,C) = 1-c_2, \; u_2(C,D) = 1,\; u_2(D,C) = 1-c_2, \; u_2(D,D) = 0\]
	A pure strategy is \[s_i: [\underline{c}, \overline{c}] \to \{C,D\}\] and player $i$'s payoff is \[u_i(s_i,s_j) = \max\{s_1,s_2\}-c_is_i\] and a BNE is a pair of strategies $(s_1^*(\cdot), s_2^*(\cdot))$ such that for each player $i$ and every possible value of $c_i$ the strategy $s_i^*(c_i)$ maximizes the expected value of $u_i$. Consider $z_j = \mathbb{P}(s_j^*(c_j) = 1)$ to be the equilibrium probability that player $j$ goes for C. The expected output for $i$ if they don't contribute then is $z_j\cdot 1$ and $1-c_j$ if the do. Hence, player $i$ will contribute if their cost is less than $1-z_j)$ which is their gain from the public good times the probability that player $j$ does not contribute. That is, the optimal strategy is $s_i^*(c_i) = 1$ if $c_i < 1-z_j$ and 0 otherwise. Thus the types of player $i$ who contributes is belong to an interval $[\underline{c},c_j^*]$ for some $c_j^*$. Since $z_j = P(c_j^*)$, the equilibrium cutoff level must satisfy $c_i^* = 1-P(c_j^*)$. Therefore $c_1^*,c_2^*$ must both satisfy the equation $c^* = 1-P(1-P(c^*))$. If there is a unique $c^*$ for this equation, then necessarily $c_i^* = c^* = 1-P(c^*)$. For instance, if $P$ is the uniform distribution on $[0,1]$, (i.e. $P(c) = \frac{c}{2})$, then $c^*$ is unique and is equal to $\frac{2}{3}$.
	\subsection{First Prize auction with two types}
	Let's consider a first-price auction with two types $\underline{\theta}$ (cheap) and $\overline{\theta}$ (generous). Let $\underline{p}, \overline{p}$ be the respective probabilities for a player $i$. Let's look for an equilibrium where $\underline{\theta}$ bids $\underline{\theta}$ and type $\overline{\theta}$ randomizes according to the continuous distribution $F(s)$ in $[\underline{s},\overline{s}]$. Clearly $\underline{s} = \underline{\theta}$. If $\underline{s} > \underline{\theta}$ then a player with type $\overline{\theta}$ would be better off bidding just above $\underline{\theta}$ rather than bidding $\underline{s}$ as this would not reduce her probability of winning ans would reduce her payment if they win. For player $i$ with type $\overline{theta}$ to play in an equilibrium with a mixed strategy with support $[\underline{s},\overline{s}$], it must hold that \[(\overline{\theta}-s)[\underline{p}+\overline{p}F(s)] = c \; \forall s \in [\underline{s},\overline{s}]\] where $c$ is constant. That is, the expected payoff of the generous type should be constant where for bid $s$ the generous tpe gets $\overline{\theta}-s$ with probability $\underline{p}+ \overline{p}F(s)$. as $F(\underline{\theta}) = 0$ the constant $c$ is equal to $(\overline{\theta}-\underline{\theta})\underline{p}$. This $F(\cdot)$ is defined by \[(\overline{\theta} - s)[\underline{p}+\overline{p}F(s)] = (\overline{p}-\underline{p}) \]
	Let $G(s) = \underline{p}+\overline{p}F(s)$ denote the cumulative distribution of bids for $s\geq \theta$. Then we get \[(\overline{\theta}-s)G(s) = (\overline{\theta}-\underline{\theta})\underline{p}\] Finally $F(s) = 1$ implies $(\overline{\theta} - \overline{s}) = (\overline{\theta} - \underline{\theta})\underline{p}$, that is \[\overline{s} = \overline{p}\overline{\theta}+\underline{p}\underline{\theta}\]
	\subsection{Double Auction}
	Here buyers and sellers make their moves simultaneously. The sellers submit an asking price and the buyers their bids. An auctioneer then chooses a price that clears the market:\\
	All sellers who asked for less sell and all the buyers who offered more buy. The number of units that are supplied at this price equals the number of units that were demanded.\\
	The simplest possible version is with one seller (player 1) and one buyer (player 2). The seller has a cost $c$ and the buyer has a valuation $v$. The simultaneously choose bids $b_1,b_2 \in [0,1]$. If $b_1\leq b_2$ they trade at price $t = \frac{b_1+b_2}{2}$. Otherwise they do not trade. We get the utility functions \[u_1 = t-c\] and \[u_2 = v-t\] if $b_1\leq b_2$. If we consider complete information, there are infinite NE (e.g. of the form $b_1=b_2 \in [c,v])$.\\
	The incomplete version is slightly more difficult. The players' types are distributed according to $P_i$ in $[0,1]$. We are looking for a Bayesian NE $s_1(\cdot)$ and $s_2(\cdot)$ depending on $c$ and $v$ respectively. Let $F_1$, $F_2$ be the equilibrium cumulative distributions of the players' types. That is $F_1(b)$ is the probability that the seller has a cost that induces her to bid less than $b$ \[F_1(b) = \mathbb{P}(s_1(c)\leq b)\]
	The expected payoff of the seller bidding $b_1$ with cost $c$ thus is \[\int_{b_1}^{1} \left(\frac{b_1+b_2}{2}-c\right)\, dF_2(b_2)\] Obviously, the probability that $b_2$ is in a given interval $[\underline{b}_2, \overline{b}_2]$ is given by $F_2(\overline{b}_2) - F_2(\underline{b}_2)$.\\
	Now, consider two costs $c',c''$ for the seller. Let $b_1' = s_1(c')$ and $b_1'' = s_1(c'')$. The optimization by the seller requires \[\int_{b_1'}^1 \left(\frac{b_1' + b_2}{2} - c'\right)\,dF_2(b_2) \geq \int_{b_1''}^1\left(\frac{b_1'' + b_2}{2} - c'\right)\,dF_2(b_2)\] that is $b_1'$ is the best response to $c'$ and the integral is the expected payoff for player 1. Since \[\int_{b_1'}^1 \left(\frac{b_1' + b_2}{2} - c''\right)\,dF_2(b_2) \leq \int_{b_1''}^1\left(\frac{b_1'' + b_2}{2} - c''\right)\,dF_2(b_2)\] We conclude \[(c'-c'')(F_2(b_1'') - F_2(b_1')) \geq 0\] so that $b_1'' \geq b_1'$ and $c' \geq c''$ and similarly for the buyer. The optimization problem of the seller thus is \[\max_{b_1} \int_{b_1}^1 \left(\frac{b_1+b_2}{2}-c\right)\, dF_2(b_2)\] This implies \begin{enumerate}
		\item $\frac{1}{2}(1-F_2(s_1(c))) - (s_1(c)-c)f_2(s_1(c)) = 0$ or 
		\item $\frac{1}{2}(1-F_2(s_1(c))) - (s_1(c)-c)f_2(s_1(c)) > 0$ and $s_1(c) = 1$ or 
		\item $\frac{1}{2}(1-F_2(s_1(c))) - (s_1(c)-c)f_2(s_1(c)) < 0$ and $s_1(c) = 0$
	\end{enumerate}
	Since $f_2(1) = 1$ and $F_2(0) = 0$, only the first one can be satisfied. We get an analogous formula for the buyer \[\max_{b_2} \int_0^{b_2} \left(v-\frac{b_1+b_2}{2}\right) \d F_1(b_1)\] leading to \[(v-s_2(v))f_1(s_2(v)) = \frac{1}{2} F_1(s_2(v))\]
	Now, assuming $P_1$, $P_2$ uniform we are looking for affine strategies \[s_1(c) = \alpha_1 + \beta_1c\]\[s_2(v) = \alpha_2 + \beta_2v\] This implies $f_i(b) = \frac{1}{\beta_i}$. Plugging in this and doing a bit of number crunching gives \[\beta_1 = \beta_2 = \frac{2}{3}, \alpha_1 = \frac{1}{4}, \alpha_2 = \frac{1}{12}\]
	\subsection{Two-Player First-Price Auction}
	Consider a two-player first-price sealed-bid auction in which the bidders' valuations on a good are iid on $[0,1]$. The bids $b_i$ are submitted simultaneously. Bidder $i$ has a valuation of $v_i$ thus earning $v_i-b_i$ in case of winning the auction. In case of a tie, the winner is chosen at random. Suppose that each bidder bids $b_i = v_ic$ where $c$ is a constant. Note that $\mathbb{P}[b_i = b_j] = 0$ given a value of $v_i$. Thus player $i$'s optimal bid solves \[\max_{b_i} ((v_i-b_i)\mathbb{P}[b_i > v_jc]) = \max_{b_i} ((v_i-b_i)\frac{b_i}{c})\] By taking the first order condition we get $\frac{v_i}{c} = 2\frac{b_i}{c}$.
	\subsection{Cournot Duopoly with Incomplete Information}
	Consider a Cournot duopoly operating in a market with inverse demand $P(Q) = a-Q$ where $Q = q_1+q_2$. Both firms have total costs $c_i(q_i) = cq_i$ but the cost is uncertain. It is high ($c = c_H$) with probability $\gamma$ and low ($c = c_L$) with probability $1-\gamma$. Information is asymmetric, that is firm 1 knows $\gamma$ but firm 2 does not. The payoff for player 1 is \[u_1(q_1,q_2,c) = \begin{cases}
		(a-Q-c_H)q_1, & \text{ if } c = c_H\\
		(a-Q-c_L)q_1, & \text{ if } c = c_L
	\end{cases}\]
	\begin{example}
		Consider a car-seller and a potential buyer. Suppose that the quality of the car $\theta$ is uniformly drawn from $[0,1]$ which is known to the seller but not to the buyer. Suppose that the buyer can make an offer $p \in [0,1]$ which the seller can accept or reject. (This can be modelled as a simultaneous move auction since the buyer can announce a range of acceptable prices.)\\
		The payoff functions are given by \[u_S = \begin{cases}
			p, & \text{ if offer accepted}\\
			\theta, & \text{ otherwise}
		\end{cases}\]
		\[u_B = \begin{cases}
			a+b\theta-p, & \text{ if offer accepted}\\
			0, & \text{ otherwise}
		\end{cases}\]
		Assume that $a \in [0,1)$, $b \in (0,2)$ and that $a+b>1$ so that it is more efficient for the buyer to own the car $\forall \theta \in [0,1]$. Give a BNE in which the seller accepts if and only if $p \geq \theta$.\\
		\underline{Solution:}\\
		Obviously the seller accepts the offer if and only if the payoff for accepting is at least as big as the payoff for rejecting, i.e. $p \geq \theta$. Since the buyer does not know $\theta$ we have to integrate for all values of $\theta$ that are possible to get the expected payoff of the buyer, i.e. \[\int_0^p (a+b\theta-p)\, d\theta = p(a+\frac{b}{2}p - p) = pa + p^2(\frac{b}{2}-1)\]
		To maximize this we need to compute the derivative \[p(b-2)+a \overset{!}{=} 0 \Leftrightarrow p = \frac{-a}{b-2}\]
	\end{example}
	\begin{example}
		Suppose that a risk-averse bidder has utility $u(\theta-b)$ if they win and $u(0)$, that is some punishment value, if they lose.\\
		Show that for second-price auctions risk averse bidders behave like risk-neutral bidders.\\
		There are two discrete types $\overline{\theta}$ with probability $\overline{p}$ and $\underline{\theta}$ with probability $\underline{p}$ for the first-price auction. Show that here, using some randomization seen in class for $\overline{\theta}$ the auctioneer has more revenue since the bidders are willing to give more.
	\end{example}
	\subsection{Perfect Bayesian Games: Signaling Games}
	There are two players, the leader $1$ and the receiver $2$. Player 1 has private information about their type and chooses action $a_1$. Player 2's type is common knowledge. They observe $a_1$ and choose $a_2$. Before the game begins, there is common knowledge about player 2's belief about player 1's type. A strategy for player 1 is a probability distribution $\sigma_1(\cdot |\theta)$ over actions $a_1$. A strategy for player 2 is a distribution $\sigma_2(\cdot | a_1)$. Type $\theta$'s payoff then is \[u_1(\sigma_1,\sigma_2,\theta) = \sum_{a_1}\sum_{a_2}\sigma_1(a_1|\theta)\sigma_2(a_2|a_1)u_1(a_1,a_2,\theta)\] while player 2's payoff is \[\sum_\theta p(\theta)\left(\sum_{a_1}\sum_{a_2}\sigma_1(a_1|\theta)\sigma_2(a_2|a_1)u_1(a_1,a_2,\theta)\right)\]
	Player 2 observes player 1's move before choosing $a_2$ ans should thus change their belief about player 1's type to an a posteriori distribution $\mu(\cdot | a_1)$ over $\Theta$. Let $\sigma_1^*$ denote player 1's strategy. Using Bayes' rule, player 2 can update their distribution $p(\cdot)$ to $\mu(\cdot | a_1)$ if they know $\sigma_1^*$ and observe $a_1$. Thus player 2 maximizes their payoff conditioned on $a_1$ for each $a_1$ where their conditional payoff to strategy $\sigma_2(\cdot | a_2)$ is \[\sum_\theta \mu(\theta|a_1)u_2(a_1,\sigma_2(\cdot | a_1),\theta) = \sum_\theta\sum_{a_2}\mu(\theta|a_1)\sigma_2(a_2|a_1)u_2(a_1,a_2,\theta)\] Therefore a Perfect Bayesian Equilibrium is a strategy profile $\sigma^*$ and posteriori beliefs $\mu(\cdot | a_1)$ s.t. \begin{enumerate}
		\item $\forall \theta$: $\sigma_1^*(\cdot | \theta) \in \arg\max_{\alpha_1} u_1(\alpha_1,\sigma_2^*,\theta)$
		\item $\forall a_1$: $\theta_2^*(\cdot | a_1) \in \arg \max_{\alpha_2} \sum_\theta \mu(\theta|\alpha_1)u_2(a_1,\alpha_2,\theta)$
		\item $\mu(\theta|a_1) = p(\theta)\sigma_1^*(a_1|\theta)/\sum_{\theta'\in \Theta} p(\theta')\sigma_1^*(a_1|\theta')$ if $\sum_{\theta' \in \Theta} p(\theta')\sigma_1^*(a_1|\theta') > 0$ and $\mu(\cdot |a_1)$ is any probability distribution on $\Theta$ if $\sum_{\theta' \in \Theta} p(\theta')\sigma_1^*(a_1|\theta') \leq 0$. 
	\end{enumerate}
	1 and 2 are the perfect conditions. 3 corresponds to the application of Bayes' rule.\\
	A pure perfect Bayesian NE where every type of player 1 play the same strategy is called a \textit{pooling BNE}. In this case player two can not change their distribution on types. A pure perfect Bayesian NE where player 1's strategy is unique to their type is called a separating NE. Here, player 2 can uniquely determine player 1's type.
	\begin{definition}[the intuitive criterion]
		Let $a_1$ be a strategy for player 1 and let $J(a_1)$ be the set of types whose equilibrium payoff is better than their payoff when playing $a_1$ and player 2 plays an undominated strategy. The equilibrium fails the intuition test if there is a type $\theta$ who would do better by playing $a_1$ than by playing the equilibrium strategy if player 2's beliefs assign probability 0 to all types in $J(a_1)$.
	\end{definition}
	\begin{example}[pronking]
		A herd of antelopes is being hunted by a lion. $\frac{2}{5}$ of the antelopes are tedious (T) and the other are fast (F). An antelope can either pronk (P) or immediately start running (R). The lion can then decide whether to hunt (H) or not (N). The payoffs are as follows 
		\begin{center}
			\begin{table}[h]
				\begin{tabular}{|c|c|c|}
					\hline
					$(a_A,a_L)$ & T & F\\
					\hline
					(P,H) & (-6,7) & (4,-5)\\
					\hline
					(P,N) & (4,0) & (9,0)\\
					\hline
					(R,H) & (-5,6) & (5,-6)\\
					\hline
					(R,N) & (5,0) & (5,0)\\
					\hline
				\end{tabular}
			\end{table}
		\end{center}
		Let $\mu(A|X)$ be the belief of the lion that an antelope who signals $X$ is of type $A$.\\
		\underline{Pooling Equilibria}: Pooling on P\\
		The lion needs to react with N otherwise both types are better of by running immediately. The lion has gained no information about the types so he react with N since \[0 \geq 7\cdot \frac{2}{5} - 5\cdot \frac{3}{5}\]
		We need to ensure that the antelopes don't deviate. This is the case if the lion reacts with H to R. Let $\mu = \mu(T|R)$. Then the lion reacts with H to R if \[6\mu - 6(1-\mu) \geq 0 \Leftrightarrow \mu \geq \frac{1}{2}\]
		Thus ((P,P),(H,N),$\frac{2}{5},\mu(T|R)\geq\frac{1}{2})$ is a pooling equilibrium.\\
		Pooling on R:
		The lion reacts with N as \[6\cdot \frac{2}{5} - 6\cdot \frac{3}{5} \leq 0\] This would give incentive to fast antelopes to deviate and pronk. Therefore we need that the lion reacts with H to P. Let $\mu = \mu(T|P)$. Then this is the case if \[7\mu - 5(1-\mu) \geq 0 \Leftrightarrow \mu \geq \frac{5}{12}\] Thus there is a second pooling equilibrium ((R,R),(H,N),$\mu(T|P)\geq \frac{5}{12},\frac{2}{3}$).\\
		\underline{separating equilibria}:\\
		There are none. Assume that T always plays P and F always plays R. Then the lion will only hunt if the antelope is tedious, i.e. if they pronked. Then type T would deviate.\\
		Assume that T always plays R and F always plays P. Then the lion will only hunt if the antelope is tedious, i.e. if they ran. Then type T would deviate.\\
		\underline{intuition criterion}:\\
		In the first pooling equilibrium, let $a_1 = R$. Then $J(R) = \{F\}$ and we should assign $\mu(F|R)=0$. As in equilibrium $\mu(T|R) \geq \frac{1}{2}$ we get $\mu(F|R)\leq \frac{1}{2}$ so the new assignment is in accordance to our equilibrium which means that it passes the intuition test.\\
		In the second pooling equilibrium, let $a_1 = P$. Then $J(P) = \{T\}$. Then we should assign $\mu(T|P) = 0$. This is a contradiction as in equilibrium $\mu(T|P) \geq \frac{5}{12}$. Therefore the second PE fails the test.
	\end{example}
	\begin{example}[Spencer's Level of Education Game]
		Player 1 (the worker) chooses a level of education $a_1 \geq 0$ and their private cost of investing in $a_1$ is $\frac{a_1}{\theta}$ where $\theta$ is the type representing the workers ability. Player 2 (i.e. the company) aims to minimize the quadratic difference of the wage $a_2$ offered to player 1 and their ability. Thus player 2 offers $a_2(a_1) = \mathbb{E}[\theta | a_1]$. Player 1's objective function is $a_2 - \frac{a_1}{\theta}$. They have two types $\theta', \theta''$ with $0<\theta' < \theta''$ and respective probabilities $p',p''$. Let $\sigma_1', \sigma_1''$ be the equilibrium strategies of type $\theta', \theta''$. Note that if $a_1' \in support(\sigma')$ and $a_1'' \in support(\sigma_1'')$ $a_1' \leq a_1''$: \[a_2(a_1') - a_1'/\theta' \geq a_2(a_1'' - a_1''/\theta')\] i.e. $a_1'$ is the best response to $\theta'$ and \[a_2(a_1'') - a_1''/\theta'' \geq a_2(a_1') - a_1'/\theta''\] Summing these up yields \[(1/\theta' - 1/\theta'')(a_1'' - a_1') \geq 0\] i.e. $a_1' \leq a_1''$.\\
		First, in a separating equilibrium, the low-ability worker gets revealed and therefore gets a wage equal to $\theta'$, the educational investment choice being $a_1' = 0$.\\
		Let $a_1''$ be the equilibrium action of type $\theta''$. In order for ($a_1' = 0, a_1''$) to be part of a SE, it must be the case that type $\theta'$ does not prefer $a_1''$ to $a_1'$: \[\theta' \geq \theta'' - a_1''/\theta'\]
		or \[a_1'' \geq \theta'(\theta''-\theta')\]
		Similarly, type $\theta''$ cannot prefer $a_1'$ to $a_1''$ i.e. \[a_1'' \leq \theta''(\theta'' - \theta')\] Hence \[\theta'(\theta'' - \theta') \leq a_1'' \leq \theta''(\theta''-\theta')\]
		Conversely, suppose that $a_1''$ belongs to this interval. Consider the beliefs \[\mu(\theta'| a_1) = 1 \text{ if } a_1 \neq a_1'' \text{ and } \mu(\theta'|a_1'') = 0\]
		Clearly, both types prefer $a_1 = 0$ to any other $a_1 \notin \{0,a_1''\}$ since any such $a_1$ yields low wage $\theta'$ anyway. Since $\theta'$ prefers 0 to $a_1''$ and $\theta''$ prefers $a_1''$ to 0, there is a continuum of SE.\\
		Now, in a PE both types choose the same action $\tilde{a}_1 = a_1' = a_1''$. The wage is then \[a_2(\tilde{a}_1) = p'\theta' + p''\theta''\] The easiest way to support this as a pooling outcome is to assign the pessimistic beliefs \[\mu(\theta'|a_1) = 1\] to any action $a_1 \neq \tilde{a}_1$ since this minimizes both types' temptation to deviate. Therefore $\tilde{a}_1$ is a PE education iff \[\theta' \leq p'\theta' + p''\theta'' - \tilde{a}_1/\theta \; \forall \theta\]
		Since $\theta' < \theta''$ the low-ability type is the most tempted to deviate to $a_1 = 0$ in order to minimize education costs. The binding constraint thus is \[\tilde{a}_1 \leq p''\theta'(\theta''-\theta')\] so there is also a continuum of pooling equilibria.\\
		\underline{intuitive criterion}: The pooling equilibrium is unreasonable since the low-ability type sine the low-ability type $\theta'$ cannot gain with a deviation from the common action $\tilde{a}_1$ while the high-ability type can improve by jumping to the SE value.
	\end{example}
	\begin{example}[Cheap-Talk Committee Signaling Game]
		The sender is a committee and the receiver is a candidate. The committee has a type $\theta$ that is uniformly distributed in $[0,1]$. First the committee chooses an action $a_1$ which the candidate observes and responds with $a_2$. The candidates payoff is \[u_2(\theta,a_1,a_2) = -(a_2-\theta)^2\] and the committees' \[u_1(\theta,a_1,a_2) = -(a_2-(\theta+b))^2\] for some fixed $b$ known by both players. Suppose all types in $[0,x_1)$ choose the same action and so do the ones in $[x_1,1]$. Let's now check which range $b \in [0,1]$ should have in order for such a semi-pooling equilibrium to exist and which value $x_1$ should have given $b$.\\
		$x_1$ should be the threshold value for which the committee is indifferent between choosing on or another attitude. Hence given the payoff function after learning the action from the types in $[0,x_1)$, the candidate's optimal action is $\frac{x_1}{2}$ while after learning the action from the types in $[x_1,1]$ it should be $\frac{x_1+1}{2}$. The types in one of the classes must always prefer the response from player 2 to their type over the other response in order do dis-incentivize deviation. Therefore for a semi-pooling equilibrium to exist, $x_1$ must be the type $\theta$ whose optimal action $\theta+b$ equals exactly the midpoint between these two actions \[x_1 = \frac{1}{2}-2b\] As the type space is $[0,1]$, $x_1$ must be positive, so a semi-pooling P.B.E. exists only if $0\leq b < \frac{1}{4}$.
	\end{example}
	\section{Perfect Bayesian Model}
	Each player $i$ has a type $\theta_i \in \Theta_i$. Let $\theta= (\theta_1,...,\theta_n)$ and assume that the types are independent of one another. Then \[p(\theta) = \prod_{i=1}^n p_i(\theta_i)\] Usually, each player knows their own type but not the type of anyone else. The game is played in periods $0,...,T$. At each period $t$ all players choose an action $a_i^t$. Then $a^t$ is the vector of stage-$t$ actions and $h^t = (a^0,...,a^{t-1})$ is the history in stage $t$. A behaviour strategy is a map of the set of possible histories and types into the action spaces with $\sigma_i(a_i|h^t,\theta_i)$ being the probability of $a_i$ given $h^t$ and $\theta_i$. We write $u_i(h^{T+1},\theta)$ for $i$'s payoff. We have an extension similar to subgame perfection but with incomplete information. We denote $i$'s belief that their opponents' type are $\theta_i$ by $\mu_i(\theta_{i-1}|\theta_i,h^T)$. What restrictions need to be made on $i$'s beliefs?
	\begin{enumerate}
		\item Posteriori beliefs must be independent and all types of player $i$ have the same beliefs. For all $\theta,t,h^t$ it is \[\mu_i(\theta_{-i}|\theta_i,h^t) = \prod_{j\neq i} \mu(\theta_j|h^t)\]
		Even unexpected observations do not make player $i$ believe that their opponents collaborated.
		\item Bayes Rule is used to update beliefs from $\mu_i(\theta_j|h^t)$ to $\mu_i(\theta_j|h^{t+1})$ whenever possible. There always exists a $\hat\theta_j$ such that $\mu_i(\hat\theta_j|h^t) > 0$.
		\item if $a_j = \hat a_j^t$ then \[\mu_i(\theta_j|(h^t,a^t)) = \mu_i(\theta_j|(h^t,\hat a^t))\] Even if player $j$ deviates at period $t$, the updating process should not be influenced by the actions of the other players. In other words, one cannot gain information about player $j$'s type from the actions of the other players.
		\item For all $h^t,\theta_k$ and $i\neq j \neq k$ it holds that \[\mu_i(\theta_k|h^t) = \mu_j(\theta_k|h^t) = \mu(\theta_k | h^t)\]
		That is, all players have the same beliefs about each other's strategies.
	\end{enumerate}
	A perfect Bayesian Equilibrium or PBE is a pair $(\sigma,\mu)$ that satisfies these four restrictions as well as the usual domination restriction.
	\begin{example}[two-player two-stage public good]
		In each of the two stages each player decides whether to contribute to the stage-$t$-good. The cost is $c_i$ in each stage which is private information. If at least one player contributes, they get payoff 1 otherwise 0. There is also a discount factor $\delta$. Both players believe that every $c_i$ is drawn independently from the same continuous strictly increasing cumulative distribution $P$ on $[0,\bar c]$ with $\bar c > 1$. If there is a unique solution to the equation $c^* = 1-P(1-P(c^*))$ then the single-period version of the game has a unique BE and $c^*$ is given by the equation $c^* = 1-P(c^*)$. Types $c_i \leq c^*$ contribute and the others do not.\\
		Now, in the repeated game, the strategy for player $i$ is $\sigma_i^0(1|c_i)$ and $\sigma_1^1(1 | h^1,c_i)$ for all histories $h^1 \in \{00,01,10,11\}$. There is a cutoff cost $\hat c_i$ for eacht player s.t. player $i$ contributes in the first stage if and only if $c_i\leq \hat c_i$.
		\begin{itemize}
			\item \underline{neither player contributed ($h^1 = 00$)}:\\
			Everyone knows that their opponents costs exceed $\hat c$. Thus \[P(c_i \leq c | 00) = \frac{P(c) - P(\hat c)}{1-P(\hat c)}\] for $c \in [\hat c,\bar c]$ and \[P(c_i \leq c | 00) = 0\] for $c \leq \hat c$. In a symmetric second-stage equilibrium each player $i$ contributes iff $\hat c \leq c_i \leq c'$. The cutoff cost is equivalent to the probability \[\frac{1-P(c')}{1-P(\hat c)}\] that the opponent does not contribute. Note that $\hat c < c' < 1$.
			\item \underline{both players contributed $(h^1 = 11)$}:\\
			The posteriori cumulative distribution is then \[P(c_i \leq c | 11) = \frac{P(c)}{P(\hat c)}\]
			for $c \in [0,\hat c]$ and \[P(c_i \leq c | 11) = 1\]
			for $c \in [\hat c \bar c]$. In a symmetric second-period equilibrium each player $i$ contributes if and only if $c_i \leq \tilde c$. Each player's cutoff cost is equal to the conditional probability that the opponent does not contribute \[\tilde c = \frac{P(\hat c) - P(\tilde c)}{P(\hat c)}\]
			\item \underline{one player contributed $h^1 = 01$ or $h^1 = 10$}:\\
			Suppose player $i$ contributed in stage 0 and $j$ did not. Hence $c_i \leq \hat c$ and $c_j \geq \hat c$. If in equilibrium during stage 1 player $i$ contributes and player $j$ does not, the second stage payoffs of type $\hat c$ are $v^{10}(\hat c) = 1 - \hat c$ and $v^{01}(\hat c) 0 1$.\\
			In the first-stage game type $\hat c$ must be indifferent about contributing or not. Therefore \[1-\hat c + \delta(P(\hat c)v^{11}(\hat c) + [1-P(\hat c)]v^{10}(\hat c)) = P(\hat c) + \delta(P(\hat c)v^{01}(\hat c) + [1-P(\hat c)]v^{00}(\hat c))\] Using the formulas for the second-period stages and some previous equation we obtain \[1-P(\hat c) = \hat c + \delta P(\hat c)\tilde c\] This defines $\hat c$.
		\end{itemize}
	\end{example}
	\begin{example}[perfect Bayesian Prisoners' Dilemma]
		We play without discounting. With probability $p$ player can play only a Tit-for-Tat strategy. That is, they mimic the other player's strategy from the previous round. Player 2 does not know 1's type.\\
		The rational type should try to convince the other player that they are Tit-for-Tat. That is, the rational type will mimic the other player's strategy (confess, as this is the NE in the game with complete information) but in the very last stage, the deviate to earn a higher reward with no possibility for punishment as the game ends. We work with the table below,
		\begin{table}[h]
			\centering
			\begin{tabular}{|c|c|c|}
				\hline
				& C & F\\
				\hline
				C & (1,1,) & (y,x)\\
				\hline
				F & (x,y) & (0,0)\\
				\hline
			\end{tabular}
		\end{table}
		where $x>1$ and $y<0$ to ensure that the game is a Prisoners' Dilemma game. $x+y<2$ ensures that the rational player plays above strategy.\\
		Consider a 2-stage game. The two types of player 1 choose different strategies in the first stage. Therefore player 2 begins stage 2 knowing the type of player 1. It is easily seen that player 2 cooperates in the first stage if \[p+(1-p)y \geq 0 \; (*)\]\\
		Now, let's move to a 3-stage game. We assume that $(*)$ always holds. If player 2 and the rational type both cooperate in the first stage, then the equilibrium path for the second and third stages are the two stages of the 2-stage game. The goal is now to find sufficient conditions for player 2 and the rational type to cooperate in the first period. Here, the payoff for the rational type is $1+x$ and the expected payoff for player 2 is $1+p+(1-p)y +px$. If the rational type betrays in the first stage, it becomes common knowledge that player 1 is rational and both players betray from here on. This is less than their equilibrium payoff of $1+x$, so they would not deviate. If player 2 deviates in the first stage, the only get $x$. Thus the given strategy is a NE if \[1+p+(1-p)y +px \geq x\] and thus given $(*)$ \[1+px \geq x \; (**)\]
		Alternatively, player 2 could deviate by betraying in the first stage but cooperating in the second. Then Tit-for-Tat would cooperate in the third stage. player 2's expected payoff then becomes $x+y+px$ which is less than the equilibrium payoff, provided that \[1+p+(1-p)y +px \geq x+y+px\; (***)\]
		For a $T$-stage repeated Prisoners' Dilemma we suppose that rational player 1 and player 2 cooperate until stage $T-2$. After that the stages become those given in the 2-stage game. If player 1 where to betray in any stage $t<T-1$, it would become common knowledge that player 1 is rational. Then player 1 would receive payoff $x$ in period $t$ and $0$ thereafter. However in equilibrium the payoff would be $(T-t-1)+x$, so betraying is not profitable for any period $t<T-1$. The argument for the 2-stage game shows that a rational player 1 would not deviate in stages $T-1$ or $T$.\\
		For player 2 a similar argument applies. The 2-stage game shows that player 2 does not deviate by cooperating until stage $T-2$ and then betrays in period $T-1$. Player 2 would also not try to deviate by cooperating in every stage except of stage $t$ given that the cooperative equilibrium will be played in the continuation game beginning at stage $t+2$, because the payoff would be $x+y+(T-(t+2)-1)+p+(1-p)y+px$ which is less than the equilibrium payoff.
	\end{example}
	\begin{example}[chain store game]
		An incumbent company faces the potential entry of another company into the market. If it stays out, the incumbent enjoys a monopoly (payoff $a>0$), if it enters it must decide whether to fight or not. If the incumbent accommodates, it will get payoff 0 and -1 if it decides to fight. The incumbents payoff will be the sum of per-stage payoffs adjusted for the discount factor $\delta$. Entrants have two types, tough or weak. A tough entrant always enters and a weak entrant has payoff 0 when staying out, -1 when entering and being fought and $b>0$ when entering and being accommodated. The entrant's type is private information and a type is tough with probability $q$ independently of others. Therefore the incumbent has a short-run incentive to accommodate, whereas a weak entrant enters only if it is expected that the probability of fighting is less than $\frac{b}{b+1}$. In a finite-horizon game the incumbent accommodates in the last stage so the least entrant enters whatever type is has. Then, in the penultimate stage, the incumbent also accommodates and the penultimate entrant also enters. By backwards induction this applies for to all stages of a finite game.\\
		Now suppose that all players' payoffs are private information an that with probability $p$ the incumbent is tough (that is payoffs are s.t. it will fight in every market along any equilibrium path) and otherwise weak. Each entrant is tough with probability $q$ independently of others. Tough entrants enter regardless of how they expect the incumbent to respond. In order to solve this for the 1-stage game, then the 2-stage game and so on.\\
		\underline{1-period game}: If there is entry, the incumbent accommodates iff the incumbent is weak. The weak entrant gets $(1-p)b-p$ from entry. Thus a weak type enters iff $p<\frac{b}{b+1} = \bar p$.
		\underline{2-period game}: The incumbent faces two different entrants in succession. Entrant 2 is faced first and entrant 1 observes the outcome in market 2 before making its own entry decision. The equilibrium depends in the prior probabilities and the parameters of the payoff functions \begin{enumerate}
			\item if $1>a\delta(1-q)$ i.e. $q>\bar q = \frac{(a\delta-1)}{a\delta}$ the maximum long-run benefit of fighting $(\delta a(1-q))$ is less than its cost (which is 1), so a weak incumbent will not fight in market 2. Since the tough incumbent will fight a weak entrant 2 enters if $p < \bar p$ and stays out if $p>\bar p$. A weak entrant 1 enters if the incumbent accommodates in market 2 and stays out if the incumbent fights.
			\item If $q<\bar q$ the weak incumbent fights in market 2 if that deters entry since accommodating reveals that the incumbent is weak and causes entry to occur. In this case, if entrant 2 enters, the weak incumbent must fight with positive probability. It cannot be an equilibrium for the weak incumbent to accommodate with probability 1 in market 2m since then if the incumbent fights the entrant believes that the type is tough and so fighting deters entry in the next period.
			\begin{enumerate}
				\item If $p>\bar p$ then since the tough incumbent always fights, the posteriori probability that the incumbent is tough given taht the choice is to fight in market 2 is at least $p$ and so fighting in market 2 deters a weak entrant in market 1. Thus, the weak incumbent fights with probability 1 in market 2, the weak entrant stays out of market 2 and the weak incumbent's expected payoff is $((1-q)a-q) + \delta(1-q)a$. 
				\item If $p<\bar p$ it is not an equilibrium for the weak incumbent to fight with a similar argument as above. Nor can it be an equilibrium for the weak incumbent with probability 1 as fighting would deter entrance and the incumbent would want to switch.
			\end{enumerate}
			Thus, in equilibrium the weak incumbent must randomize, which requires that when the incumbent fights in market 2 the weak entrant 1 randomizes in a way that makes the weak incumbent indifferent in market 2.\\
			By letting $\beta$ be the conditional probability that a weak incumbent fights entry in market 2 and a recall that the tough incumbent fights with probability 1. By Bayes' rule \[\mathbb{P}[tough|fight] = \frac{p}{p+\beta(1-p)}\] and for this to equal $\bar p, \beta$ must equal $\frac{p}{(1-p)b}$. The total probability that entry in market 2 is fought is \[p+(1-p)(\frac{p}{(1-p)b}) = \frac{p(b+1)}{b}\] so the weak entrant will stay out of the market 2 if this probability surpasses the critical level, that is if \[\frac{p(b+1)}{b} > \frac{b}{b+1}\] or alternatively if $p>\left(\frac{b}{b+1}\right)^2 = \bar p^2$. In this case the weak incumbent's expected average payoff is positive instead of 0f or the same parameters in the one-entrant game. If $p<\bar p^2$ the weak entrant enters market 2 and the weak incumbent's payoff is 0.
		\end{enumerate}
		\underline{3-stage game}: If $p>\bar p^2$, the weak incumbent is certain to fight in market 3 ant the weak entrant stays out. If $p$ is between $\bar p^3$ and $\bar p^2$, the weak incumbent randomizes an the weak entrant stays out. If $p<\bar p^3$ the week incumbent randomizes and the weak entrant enters. 
		\underline{$n$-stage game}: For a fixed $p$ and $n$ entrants the weak entrant stays out until the first period $k$ where $p<\left(\frac{b}{b+1}\right)^k$. So for the first $n-k$ periods, the incumbent hast payoff $a(1-q)-q$ per period. If $\delta=1$ we get the following unique equilibrium:
		\begin{enumerate}
			\item If $q>\frac{a}{a+1}$ then the weak incumbent accommodates at the first entry which occurs (at the latest) the first time the entrant is tough. Hence as the number of markets $n$ tends to infinity the incumbent's average payoff per period goes to 0.
			\item If $q<\frac{a}{a+1}$ then for every $p$ there is a number $n(p)$ so that if there are more than $n(p)$ markets remaining, the weak incumbent's strategy is to fight with probability 1. Thus, weak entrants stay out when there are more than $n(p)$ markets remaining.
		\end{enumerate}
		Thus $a(1-q)-q$ is the expected payoff when always fighting. So the incumbent can choose between two different commitments: to always fight and get an expected payoff of $a(1-q)-q$ or to always accommodate with expected payoff 0.\\
		If $a(1-q)>q$ then the incumbent would do better by fighting with the smallest probability that deters entry which is $\frac{b}{b+1}$. This yields an expected payoff of \[a(1-q)-q\frac{b}{b+1}\] which is better than $a(1-q)-q$ from fighting with probability 1.
	\end{example}
	\subsection{First price sealed bid auctions with two periods}
	Suppose that there are three bidders $a,b,c$ whose valuation is uniformly distributed in $[0,1]$. In the first round $a$ and $b$ fight for an item $i_1$ in which $c$ is not interested. Then in period 2 the loser of the first round fights with $c$ for item $i_2$. What would be a PBNE here?\\
	Supposing that in the first auction $a$ and $b$ bid symmetrically we look at the second auction where $c$ is competing with the loser of the first auction. W.l.o.g. we can assume that is player $a$. Is has thus become common knowledge that $a$'s valuation is less than $b^{-1}(p_1)<1$, where $p_1$ was the price paid in the first auction by $b$. That is, $c$ knows that $a$'s valuation is uniformly distributed in \[0,b^{-1}(p_1)\] while $c$'s valuation is uniformly distributed on $[0,1]$. We know that this has an solution (similar to the exercises) given by \[b_a(v_a) = \frac{v_a}{1+\sqrt{1-tv_a^2}}\] and \[b_c(v_c) = \frac{v_c}{1+\sqrt{1+tv_c^2}}\] where $t = \frac{1}{(b^{-1}(p_1))^2}-1$.
	What is the expected utility of a player losing compared to winning in the first auction?\\
	In equilibrium the player should be indifferent, thus: \begin{itemize}
		\item in case of losing we know that $u_a^2(v_a) = \frac{v_a}{1-h\sqrt{1-v_a^2}}$ whereas $u_a^1(x) = F_b(x)(v_a-b(x)) + (1-F_b(x))\mathbb{E}[u_a^2(v_a,b(t))] = x(v_a-b(x)) + \int_{x}^{1} u_a^2(v_a,b(t))\,dt$. This needs to be maximized by $\frac{d}{dx}u_a^1(x)|_{x=v_a} = 0$ \[v_a - b(v_a-v_a)b'(v_a) - u_a^2(v_a,b(v_a)) = 0\] and by using the boundary condition $b(0) = 0$ we get \[u_a^2(v_a,b(v_a)) = \frac{v_a^2}{1+v_a} = v_a-b(v_a)-v_a b'(v_a)\] and hence $b(v) = 1-\frac{ln(1+v)}{v}$.
	\end{itemize}
	\begin{lemma}
		In a $n$-player first price auction game, each player bids \[\beta(v) = v- \int_{\underline{v}}^{v}\left(\frac{F(x)}{F(v)}\right)^{n-1} \; dx\] in equilibrium.
	\end{lemma}
	\begin{theorem}[Revenue Equivalence Theorem]
		Suppose $n$ bidders have valuations $v_1,...,v_n$ identically independently distributed with cumulative function $F(\cdot)$. Then any equilibrium of any auction game in which \begin{enumerate}
			\item the bidder with the highest valuation wins
			\item the bidder with valuation $\underline{v}$ (the minimal valuation) gets zero profit
		\end{enumerate}
		then different auction types generate the same expected payoff.
	\end{theorem}
	\subsection{Coasian Dynamics}
	A single seller faces a continuum of infinitesimal buyers. The seller cannot distinguish between the buyers and only observes the measures of the sets who accept and reject. The seller and the buyers are infinitely-lived and the bargaining process has $T 0 \infty$. The buyers valuations are uniformly distributed on $[0,1]$. Let's find a equilibrium with the following properties: \begin{enumerate}
		\item If $m^t$ is offered at date $t$ then the types $v \geq \lambda m^t$ where $\lambda > 1$ buy if they have not bought before. The other types do not.
		\item If at some date $t$ types greater than $\kappa$ have purchased before and types less than $\kappa$ have not (so that the seller's posteriori beliefs are represented by the truncated uniform distribution on $[0,\kappa]$), then the seller charges $m^t(\kappa) = \gamma \kappa$ where $0<\gamma <1$.
	\end{enumerate}
	Let $U_s(\kappa)$ be the seller's discounted value of profits when the posteriori beliefs are uniform on $[0,\kappa]$. From dynamic programming, $U_s(\cdot)$ must satisfy \[U_s(\kappa) = \max_m(\kappa - \lambda m)m + \delta U_s(\lambda m)\]
	The term $(\kappa - \lambda m)m$ is the fraction of population that will accept the offer $m \leq \frac{\kappa}{\lambda}$ and $U_s(\lambda m)$ is the continuation discounted value of profits. If $U_s$ is assumed to be differentiable the maximization with respect to $m$ yields $\kappa - 2 \lambda , + \delta U_s'(\lambda m) \overset{!}{=} 0$. On the other hand the envelope theorem can be applied \[U_s'(\kappa) = m(\kappa) = \gamma \kappa\] Substituting this yields \[1-2\lambda m + \delta \lambda^2 \gamma^2 = 0\]
	Looking at the buyers' optimization, for type $\lambda m$ to be indifferent between accepting $m$ and waiting one period to buy at the price $\lambda m$ it must be the case that \[\lambda m - m = \delta (\lambda m - \gamma \lambda m)\] or equivalently \[\lambda -1 = \delta \lambda (1-\gamma)\] which gives \[\lambda = \frac{1}{\sqrt{1-\delta}} \text{ and } \gamma = \frac{\sqrt{1-\delta} - (1-\delta)}{\delta}\]\\
	Now consider that the buyer accepts price $m^t$ at date $t$ when the buyer has valuation $v$ in a game with an arbitrary number of periods. Show the following skimming (or cutoff rule) property: the buyer accepts the price $m^t$ with probability 1 when he has valuation $v^t>v$. Let $h^t = (m^0,..., m^{t-1})$ be the history at date $t$. Type $v$ accepts $m^t$ iff \[v-m^t \geq \delta U_b(v,(h^t,m^t))\] or \[v-m^t \geq \mathbb{E}\left[\sum_{i=1}^{T-t} \delta^i(v-m^{t+i}(h^{t+i})) I^{t+i}(h^{t+i},m^{t+i},v) | (h^t,m^t)\right]\]
	where $U_b(v,(h^t,m^t))$ is the continuation valuation of the type $v$ and where $I^{t+i}(h^{t+i},m^{t+i},v)|(h^t,m^t)$ is an indicator function indicating whether type $v$ buys $(I=1)$ or not $(I=0)$ at price $m^{t+i}(h^{t+i})$ at date $t+i$.\\
	Since the expected discounted volume of trade is always less than  1 and because type $v'$ can mimic $v'$'s bargaining strategy and conversely \[\left|U_b(v',(h^t,m^t)) - U_b(v,(h^t,m^t))\right| \leq \left|v'-v\right|\]
	Therefore, for $v'>v$, \[v'-m^t-\delta U_b(v',(h^t,m^t)) \geq (v'-v) - \delta(U_b(v',(h^t,m^t)) - U_b(v,(h^t,m^t))) > 0\]
\end{document}
\documentclass[a4paper, 12pt]{article}

\usepackage{fullpage}
\usepackage[utf8]{inputenc}
\usepackage[english]{babel}
\usepackage{amsmath,amssymb}
\usepackage[explicit]{titlesec}
\usepackage{ulem}
\usepackage[onehalfspacing]{setspace}
\usepackage{amsthm}

\theoremstyle{plain}
\newtheorem{theorem}{Theorem}[section] % reset theorem numbering for each chapter

\theoremstyle{definition}
\newtheorem{definition}[theorem]{Definition} % definition numbers are dependent on theorem numbers
\theoremstyle{lemma}
\newtheorem{lemma}[theorem]{Lemma}

\theoremstyle{remark}
\newtheorem{remark}[theorem]{Remark}

\theoremstyle{corollary}
\newtheorem{corollary}[theorem]{Corollary}

\theoremstyle{example}
\newtheorem{example}[theorem]{Example}

\titleformat{\subsection}
{\small}{\thesubsection}{1em}{\uline{#1}}
\begin{document}
	\begin{titlepage} 
		\title{Topologie Zusammenfassung}
		\clearpage\maketitle
		\thispagestyle{empty}
	\end{titlepage}
	\tableofcontents
	\newpage
	\section{Topologische Räume}
	\begin{definition}
		Sei $X\neq \varnothing$. Eine Topologie auf $X$ ist eine Familie $\mathcal{T} \subset \mathcal{P}(X)$ mit \begin{enumerate}
			\item $\varnothing \in \mathcal{T}, X \in \mathcal{T}$
			\item Die Vereinigung beliebig vieler Teilmengen aus $\mathcal{T}$ ist in $\mathcal{T}$
			\item Der Durchschnitt endlich vieler Teilmengen von $\mathcal{T}$ ist in $\mathcal{T}$
		\end{enumerate}
		Das Paar $(X,\mathcal{T})$ heißt topologischer Raum, die Elemente aus $\mathcal{T}$ heißen offene Mengen.
	\end{definition}
	\begin{definition}
		Auf $X \neq \varnothing$ seien $\mathcal{T}$ und $\mathcal{T}'$ zwei Topologien mit $\mathcal{T}\subset \mathcal{T}'$. Dann heißt $\mathcal{T}'$ die feinere und $\mathcal{T}$ die gröbere Topologie. Ist zusätzlich $\mathcal{T} \neq \mathcal{T}'$, dann heißen die Relationen strikt feiner bzw. gröber.
	\end{definition}
	\begin{definition}
		Sei $X\neq \varnothing$. Eine Basis für eine Topologie auf $X$ ist eine Familie $\mathcal{B}\subset \mathcal{P}(X)$ welche die Eigenschaften erfüllt \begin{enumerate}
			\item $\forall x \in X \, \exists B \in \mathcal{B}$ sodass $x \in B$
			\item Sind $B_1,B_2 \in \mathcal{B}$ und $x \in B_1\cap B_2$, dann existiert $B_3 \in \mathcal{B}$ mit \[x \in B_3 \subset B_1\cap B_2\]
		\end{enumerate}
		Ist eine Basis $\mathcal{B}$ gegeben, so heißt $\mathcal{T}(\mathcal{B})$ die von $\mathcal{B}$ erzeugte Topologie über \[U \in \mathcal{T}(\mathcal{B}) : \Leftrightarrow \forall x \in U \exists B \in \mathcal{B} \text{ mit } x \in B \subset U\]
	\end{definition}
	\begin{theorem}
		Sei $X \neq \varnothing$, $\mathcal{B} \subset \mathcal{P}(X)$ eine Basis für eine Topologie auf $X$. Dann ist $\mathcal{T}(\mathcal{B})$ in der Tat eine Topologie.
	\end{theorem}
	\begin{theorem}
		Sei $X \neq \varnothing$ und $\mathcal{B}$ eine Basis für eine Topologie auf $X$. Dann besteht $\mathcal{T}(\mathcal{B})$ aus genau den Mengen, die sich als Vereinigung von Mengen aus $\mathcal{B}$ schreiben lassen.
	\end{theorem}
	\begin{theorem}
		Sei $(X,\mathcal{T})$ ein topologischer Raum. Sei dann $\mathcal{C} \subset \mathcal{T}$ mit der Eigenschaft \[\forall U \in \mathcal{T} \text{ und } x \in U \, \exists C \in \mathcal{C}: \; x \in C \subset U\]
		Dann ist $\mathcal{C}$ bereits eine Basis für $\mathcal{T}$.
	\end{theorem}
	\begin{theorem}
		Seien $\mathcal{B}, \mathcal{B}'$ Basen zweier Topologien $\mathcal{T}, \mathcal{T}'$. Dann sind die folgenden beiden Aussagen äquivalent \begin{enumerate}
			\item $\mathcal{T}'$ ist feiner als $\mathcal{T}$
			\item $\forall x \in X$ und $\forall B \in \mathcal{B}$ mit $x\in B$ existiert $B'\in \mathcal{B}'$ mit $x\in B' \subset B$.
		\end{enumerate}
	\end{theorem}
	\begin{theorem}
		Sei $X \neq \varnothing$. Ist dann $\mathcal{S}\subset \mathcal{P}(X)$, so gibt es eine eindeutige gröbste Topologie, welche $\mathcal{S}$ enthält. Diese sei mit $\mathcal{T}(\mathcal{S})$ notiert.\\
		Ist außerdem $\bigcup_{S \in \mathcal{S}} S = X$, dann ist $\mathcal{T}(\mathcal{S})$ genau die Menge beliebiger Vereinigungen von endlichen Durchschnitten von Mengen aus $\mathcal{S}$. $\mathcal{S}$ heißt Subbasis von $\mathcal{T}(\mathcal{S})$.
	\end{theorem}
	\section{Konstruktion Topologischer Räume}
	\subsection{Die Spurtopologie}
	\begin{definition}
		Sei $(X,\mathcal{T})$ ein topologischer Raum und sei $\varnothing \neq Y \subset X$. Dann ist \[\mathcal{T}_Y = \{U \cap Y: U \in T\}\] die Spurtopologie.
	\end{definition}
	\subsection{Produkttopologie}
	\begin{definition}
		Seien $X,Y$ topologische Räume. Dann ist die Menge \[\mathcal{B} = \{U\times V: U \text{ offen in } X, V \text{ offen in } Y\}\] Basis einer Topologie (der Produkttopologie) auf $X\times Y$. $\mathcal{B}$ selber ist aber keine Topologie.
	\end{definition}
	\begin{theorem}
		Ist $\mathcal{B}$ eine Basis für eine Topologie auf $X$ und ist $\mathcal{C}$ eine Basis für eine Topologie auf $Y$, so ist \[\mathcal{D} 0 \{B\times C: B \in \mathcal{B}, C \in \mathcal{C}\}\] eine Basis für die Produkttopologie auf $X\times Y$.
	\end{theorem}
	\begin{theorem}
		Es seien $X,Y$ topologische Räume und $A\subset X, B\subset Y$. Dann ist die Produkttopologie der Spurtopologien bezüglich $A$ und $B$ gleich der Spurtopologie der Produkttopologie bezüglich $A\times Y$ auf $X\times Y$.
	\end{theorem}
	\subsection{Quotiententopologie}
	Wir schreiben $X' = X/\sim$ als die Quotientenmenge von $X$ bezüglich der Äquivalenzrelation $\sim$. $\varphi: X \to X'$ sei die Funktion, die jedem Element $x \in X$ seinen Repräsentanten in $X'$ zuordnet.
	\begin{theorem}
		Sei $\mathcal{T}$ eine Topologie auf $X$. Dann ist \[\mathcal{T}' = \{U \in X': \varphi^{-1}(U) \in \mathcal{T}\}\] eine Topologie auf $X'$.
	\end{theorem}
	Dieses Konzept lässt sich auf allgemeine surjektive Funktionen verallgemeinern.
	\section{Inneres, Häufungspunkte, Abschluss, Grenzwerte}
	\begin{theorem}
		$A\subset Y$ ist abgeschlossen in der Spurtopologie, genau dann, wenn $A = F \cap Y$ mit $F$ in $X$ abgeschlossen.
	\end{theorem}
	\begin{theorem}
		Ist $Y\subset Y$ selber abgeschlossen, in $X$, so ist jede relativ abgeschlossene Menge $Y\subset Y$ auch abgeschlossen in $X$
	\end{theorem}
	\begin{theorem}
		Sei $X$ ein topologischer Raum und $A \subset X$. Dann gilt \begin{itemize}
			\item $x \in \bar{A} \Leftrightarrow $ für jede Umgebung $U$ von $x$ gilt $A \cap U \neq \varnothing$.
			\item ist $\mathcal{B}$ eine Basis für diese Topologie, so gilt \[x \in A \Leftrightarrow A \cap B \neq \varnothing \; \forall B \in \mathcal{B}: x \in B\]
		\end{itemize}
	\end{theorem}
	\begin{definition}
		Ein Punkt $x \in X$ heißt Häufungspunkt von $A$, wenn für jede Umgebung $U$ von $x$ gilt \[U \cap (A \setminus \{x\}) \neq \varnothing\] Das bedeutet, $x$ liegt im Abschluss von $A \setminus \{x\}$.	
	\end{definition}
	\begin{theorem}
		Sei $A'$ die Menge aller Häufungspunkte, so gilt \[\bar{A} = A \cup A'\]
	\end{theorem}
	\begin{definition}
		Eine Folge $x_n \subset X$ heißt konvergent gegen $x$, wenn es für alle offenen Umgebungen $U$ von $x$ ein $N$ gibt, sodass $x_n \in U$ für alle $n \geq N$. 
	\end{definition}
	\begin{definition}
		Ein topologischer Raum $X$ heißt Hausdorffsch, wenn $\forall x,y \in X$ gibt es zwei offene Mengen $U_x, U_y$ mit $x \in U_x, y \in U_y$ und $U_x \cap U_y \neq \varnothing$.
	\end{definition}
	\begin{theorem}
		Sei $X$ ein Hausdorff-Raum. Es gilt \begin{itemize}
			\item Alle einelementigen und alle endlichen Teilmengen von $X$ sind abgeschlossen
			\item ist $A \subset X$, dann ist $x \in X$ Häufungspunkt von $A$ genau dann, wenn jede Umgebung von $x$ unendlich viele Elemente aus $A$ enthält
			\item jede Folge aus $X$ besitzt maximal einen Grenzwert.
		\end{itemize}
	\end{theorem}
	\section{Stetigkeit}
	\begin{definition}
		Seien $X$ und $Y$ topologische Räume. Eine Abbildung $f:X \to Y$ ist stetig, wenn für alle offenen Mengen $U \subset Y$ das Urbild $f^{-1}(U)$ offen in $X$ ist.
	\end{definition}
	\begin{theorem}
		Seien $X$, $Y$ topologische Räume. Es sei $f:X \to Y$ eine Funktion, dann sind die folgenden Eigenschaften äquivalent. \begin{enumerate}
			\item $f$ ist stetig
			\item Für jede Menge $A \subset X$ ist $f(\overline{A}) \subseteq \overline{f(A)}$
			\item Für jede in $Y$ abgeschlossene Menge $F \subset Y$ ist $f^{-1}(Y)$ abgeschlossen in $X$
			\item zu jedem $x \in X$ und jeder Umgebung $V$ von $f(x)$ gibt es eine Umgebung $U$ von $x$, sodass $f(U) \subseteq V$.
		\end{enumerate}
	\end{theorem}
	\begin{definition}
		Eine bijektive Abbildung $f:X\to Y$ sodass $f$ und $f^{-1}$ stetig sind, heißt Homöomorphismus.
	\end{definition}
	\begin{definition}
		Ist $f$ eine stetige, injektive aber nicht surjektive Abbildung, so sei $\overline{Z} = f(X) \subset Y$. Die Einschränkung \[\tilde{f}: X \to \overline{Z}\] ist bijektiv. Ist $\tilde{f}$ ein Homöomorphismus, so nennen wir $f$ eine Einbettung von $X$ in $Y$.
	\end{definition}
	\begin{definition}
		Existiert zwischen $X$ und $Y$ ein Homöomorphismus, so heißen $X$ und $X$ homöomorph, geschrieben $X \cong Y$.
	\end{definition}
	\begin{theorem}
		Seien $X,Y,Z$ topologische Räume. Dann gilt \begin{enumerate}
			\item jede konstante Funktion $f:X\to Y$ ist stetig
			\item ist $A\subset X$, so ist die Inklusion $f: A \to X$ stetig
			\item sind $f:X\to Y$ stetig und $g:Y \to Z$ stetig, so ist $g\circ f$ stetig
			\item ist $f:X\to Y$ stetig und $A\subset X$, so ist die Restriktion $f:X \to A$ stetig bzgl. der Spurtopologie.
		\end{enumerate}
	\end{theorem}
	\begin{theorem}
		Seien $X,Y$ topologische Räume und es gelte $X = A\cup B$ mit $A,B$ abgeschlossen. Weiter seien die beiden Funktionen $f:A\to Y$ und $g:B \to Y$ stetig bzgl. der jeweiligen Spurtopologien und sei \[f(x) = g(x) \; \forall x \in A \cap B\] dann ist \[h(x) = \begin{cases}
			f(x), & x \in A\\
			g(x), & x \in B
		\end{cases}\] stetig auf $X$.
	\end{theorem}
	\begin{theorem}
		Es sei \[f:A \to X\times Y\] eine vektorwertige Funktion. $f$ ist stetig genau dann, wenn $f_1$ und $f_2$ stetig sind.
	\end{theorem}
	\section{Zusammenhängende topologische Räume}
	\begin{definition}
		Ein topologischer Raum $X$ heißt unzusammenhängend, wenn $\exists U,V$ offen, sodass $U\cap V = \varnothing$, $U \neq \varnothing, V \neq \varnothing$ und $X = U\cup V$. $X$ heißt zusammenhängend, wenn es keine solche Separation gibt. Das ist äquivalent dazu, dass $X$ und $\varnothing$ die einzigen abgeschloffenen Mengen sind.
	\end{definition}
	\begin{lemma}
		Sei $X$ ein topologischer Raum und sei $Y\subset X$. Eine Separation von $Y$ bzgl. der Spurtopologie ist ein Paar $(A,B)$ mit \[A\neq \varnothing, B \neq \varnothing, A\cup B = Y, \overline{A}\cap B = \varnothing, A \cap \overline{B} = \varnothing\]
	\end{lemma}
	\begin{lemma}
		Sei $X$ ein unzusammenhängender topologischer Raum mit Separation $(U,V)$. Ist $Y\subset X$ zusammenhängend, so ist $Y\subset U$ oder $Y\subset V$. 
	\end{lemma}
	\begin{theorem}
		Die Vereinigung zusammenhängender Räume ist zusammenhängend, wenn alle Räume mindestens einen Punkt gemeinsam haben.
	\end{theorem}
	\begin{theorem}
		Sind $X,Y$ zusammenhängende Räume, so ist $X\times Y$ zusammenhängend.
	\end{theorem}
\end{document}
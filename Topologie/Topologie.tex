\documentclass[a4paper, 12pt]{article}

\usepackage{fullpage}
\usepackage[utf8]{inputenc}
\usepackage[english]{babel}
\usepackage{amsmath,amssymb}
\usepackage[explicit]{titlesec}
\usepackage{ulem}
\usepackage[onehalfspacing]{setspace}
\usepackage{amsthm}

\theoremstyle{plain}
\newtheorem{theorem}{Satz}[section] % reset theorem numbering for each chapter

\theoremstyle{definition}
\newtheorem{definition}[theorem]{Definition} % definition numbers are dependent on theorem numbers
\theoremstyle{lemma}
\newtheorem{lemma}[theorem]{Lemma}

\theoremstyle{remark}
\newtheorem{remark}[theorem]{Bemerkung}

\theoremstyle{corollary}
\newtheorem{corollary}[theorem]{Korollar}

\theoremstyle{example}
\newtheorem{example}[theorem]{Beispirl}

\titleformat{\subsection}
{\small}{\thesubsection}{1em}{\uline{#1}}
\begin{document}
	\begin{titlepage} 
		\title{Topologie Zusammenfassung}
		\clearpage\maketitle
		\thispagestyle{empty}
	\end{titlepage}
	\tableofcontents
	\newpage
	\section{Topologische Räume}
	\begin{definition}
		Sei $X\neq \varnothing$. Eine Topologie auf $X$ ist eine Familie $\mathcal{T} \subset \mathcal{P}(X)$ mit \begin{enumerate}
			\item $\varnothing \in \mathcal{T}, X \in \mathcal{T}$
			\item Die Vereinigung beliebig vieler Teilmengen aus $\mathcal{T}$ ist in $\mathcal{T}$
			\item Der Durchschnitt endlich vieler Teilmengen von $\mathcal{T}$ ist in $\mathcal{T}$
		\end{enumerate}
		Das Paar $(X,\mathcal{T})$ heißt topologischer Raum, die Elemente aus $\mathcal{T}$ heißen offene Mengen.
	\end{definition}
	\begin{definition}
		Auf $X \neq \varnothing$ seien $\mathcal{T}$ und $\mathcal{T}'$ zwei Topologien mit $\mathcal{T}\subset \mathcal{T}'$. Dann heißt $\mathcal{T}'$ die feinere und $\mathcal{T}$ die gröbere Topologie. Ist zusätzlich $\mathcal{T} \neq \mathcal{T}'$, dann heißen die Relationen strikt feiner bzw. gröber.
	\end{definition}
	\begin{definition}
		Sei $X\neq \varnothing$. Eine Basis für eine Topologie auf $X$ ist eine Familie $\mathcal{B}\subset \mathcal{P}(X)$ welche die Eigenschaften erfüllt \begin{enumerate}
			\item $\forall x \in X \, \exists B \in \mathcal{B}$ sodass $x \in B$
			\item Sind $B_1,B_2 \in \mathcal{B}$ und $x \in B_1\cap B_2$, dann existiert $B_3 \in \mathcal{B}$ mit \[x \in B_3 \subset B_1\cap B_2\]
		\end{enumerate}
		Ist eine Basis $\mathcal{B}$ gegeben, so heißt $\mathcal{T}(\mathcal{B})$ die von $\mathcal{B}$ erzeugte Topologie über \[U \in \mathcal{T}(\mathcal{B}) : \Leftrightarrow \forall x \in U \exists B \in \mathcal{B} \text{ mit } x \in B \subset U\]
	\end{definition}
	\begin{theorem}
		Sei $X \neq \varnothing$, $\mathcal{B} \subset \mathcal{P}(X)$ eine Basis für eine Topologie auf $X$. Dann ist $\mathcal{T}(\mathcal{B})$ in der Tat eine Topologie.
	\end{theorem}
	\begin{theorem}
		Sei $X \neq \varnothing$ und $\mathcal{B}$ eine Basis für eine Topologie auf $X$. Dann besteht $\mathcal{T}(\mathcal{B})$ aus genau den Mengen, die sich als Vereinigung von Mengen aus $\mathcal{B}$ schreiben lassen.
	\end{theorem}
	\begin{theorem}
		Sei $(X,\mathcal{T})$ ein topologischer Raum. Sei dann $\mathcal{C} \subset \mathcal{T}$ mit der Eigenschaft \[\forall U \in \mathcal{T} \text{ und } x \in U \, \exists C \in \mathcal{C}: \; x \in C \subset U\]
		Dann ist $\mathcal{C}$ bereits eine Basis für $\mathcal{T}$.
	\end{theorem}
	\begin{theorem}
		Seien $\mathcal{B}, \mathcal{B}'$ Basen zweier Topologien $\mathcal{T}, \mathcal{T}'$. Dann sind die folgenden beiden Aussagen äquivalent \begin{enumerate}
			\item $\mathcal{T}'$ ist feiner als $\mathcal{T}$
			\item $\forall x \in X$ und $\forall B \in \mathcal{B}$ mit $x\in B$ existiert $B'\in \mathcal{B}'$ mit $x\in B' \subset B$.
		\end{enumerate}
	\end{theorem}
	\begin{theorem}
		Sei $X \neq \varnothing$. Ist dann $\mathcal{S}\subset \mathcal{P}(X)$, so gibt es eine eindeutige gröbste Topologie, welche $\mathcal{S}$ enthält. Diese sei mit $\mathcal{T}(\mathcal{S})$ notiert.\\
		Ist außerdem $\bigcup_{S \in \mathcal{S}} S = X$, dann ist $\mathcal{T}(\mathcal{S})$ genau die Menge beliebiger Vereinigungen von endlichen Durchschnitten von Mengen aus $\mathcal{S}$. $\mathcal{S}$ heißt Subbasis von $\mathcal{T}(\mathcal{S})$.
	\end{theorem}
	\section{Konstruktion Topologischer Räume}
	\subsection{Die Spurtopologie}
	\begin{definition}
		Sei $(X,\mathcal{T})$ ein topologischer Raum und sei $\varnothing \neq Y \subset X$. Dann ist \[\mathcal{T}_Y = \{U \cap Y: U \in T\}\] die Spurtopologie.
	\end{definition}
	\subsection{Produkttopologie}
	\begin{definition}
		Seien $X,Y$ topologische Räume. Dann ist die Menge \[\mathcal{B} = \{U\times V: U \text{ offen in } X, V \text{ offen in } Y\}\] Basis einer Topologie (der Produkttopologie) auf $X\times Y$. $\mathcal{B}$ selber ist aber keine Topologie.
	\end{definition}
	\begin{theorem}
		Ist $\mathcal{B}$ eine Basis für eine Topologie auf $X$ und ist $\mathcal{C}$ eine Basis für eine Topologie auf $Y$, so ist \[\mathcal{D} 0 \{B\times C: B \in \mathcal{B}, C \in \mathcal{C}\}\] eine Basis für die Produkttopologie auf $X\times Y$.
	\end{theorem}
	\begin{theorem}
		Es seien $X,Y$ topologische Räume und $A\subset X, B\subset Y$. Dann ist die Produkttopologie der Spurtopologien bezüglich $A$ und $B$ gleich der Spurtopologie der Produkttopologie bezüglich $A\times Y$ auf $X\times Y$.
	\end{theorem}
	\subsection{Quotiententopologie}
	Wir schreiben $X' = X/\sim$ als die Quotientenmenge von $X$ bezüglich der Äquivalenzrelation $\sim$. $\varphi: X \to X'$ sei die Funktion, die jedem Element $x \in X$ seinen Repräsentanten in $X'$ zuordnet.
	\begin{theorem}
		Sei $\mathcal{T}$ eine Topologie auf $X$. Dann ist \[\mathcal{T}' = \{U \in X': \varphi^{-1}(U) \in \mathcal{T}\}\] eine Topologie auf $X'$.
	\end{theorem}
	Dieses Konzept lässt sich auf allgemeine surjektive Funktionen verallgemeinern.
	\section{Inneres, Häufungspunkte, Abschluss, Grenzwerte}
	\begin{theorem}
		$A\subset Y$ ist abgeschlossen in der Spurtopologie, genau dann, wenn $A = F \cap Y$ mit $F$ in $X$ abgeschlossen.
	\end{theorem}
	\begin{theorem}
		Ist $Y\subset Y$ selber abgeschlossen, in $X$, so ist jede relativ abgeschlossene Menge $Y\subset Y$ auch abgeschlossen in $X$
	\end{theorem}
	\begin{theorem}
		Sei $X$ ein topologischer Raum und $A \subset X$. Dann gilt \begin{itemize}
			\item $x \in \bar{A} \Leftrightarrow $ für jede Umgebung $U$ von $x$ gilt $A \cap U \neq \varnothing$.
			\item ist $\mathcal{B}$ eine Basis für diese Topologie, so gilt \[x \in A \Leftrightarrow A \cap B \neq \varnothing \; \forall B \in \mathcal{B}: x \in B\]
		\end{itemize}
	\end{theorem}
	\begin{definition}
		Ein Punkt $x \in X$ heißt Häufungspunkt von $A$, wenn für jede Umgebung $U$ von $x$ gilt \[U \cap (A \setminus \{x\}) \neq \varnothing\] Das bedeutet, $x$ liegt im Abschluss von $A \setminus \{x\}$.	
	\end{definition}
	\begin{theorem}
		Sei $A'$ die Menge aller Häufungspunkte, so gilt \[\bar{A} = A \cup A'\]
	\end{theorem}
	\begin{definition}
		Eine Folge $x_n \subset X$ heißt konvergent gegen $x$, wenn es für alle offenen Umgebungen $U$ von $x$ ein $N$ gibt, sodass $x_n \in U$ für alle $n \geq N$. 
	\end{definition}
	\begin{definition}
		Ein topologischer Raum $X$ heißt Hausdorffsch, wenn $\forall x,y \in X$ gibt es zwei offene Mengen $U_x, U_y$ mit $x \in U_x, y \in U_y$ und $U_x \cap U_y \neq \varnothing$.
	\end{definition}
	\begin{theorem}
		Sei $X$ ein Hausdorff-Raum. Es gilt \begin{itemize}
			\item Alle einelementigen und alle endlichen Teilmengen von $X$ sind abgeschlossen
			\item ist $A \subset X$, dann ist $x \in X$ Häufungspunkt von $A$ genau dann, wenn jede Umgebung von $x$ unendlich viele Elemente aus $A$ enthält
			\item jede Folge aus $X$ besitzt maximal einen Grenzwert.
		\end{itemize}
	\end{theorem}
	\section{Stetigkeit}
	\begin{definition}
		Seien $X$ und $Y$ topologische Räume. Eine Abbildung $f:X \to Y$ ist stetig, wenn für alle offenen Mengen $U \subset Y$ das Urbild $f^{-1}(U)$ offen in $X$ ist.
	\end{definition}
	\begin{theorem}
		Seien $X$, $Y$ topologische Räume. Es sei $f:X \to Y$ eine Funktion, dann sind die folgenden Eigenschaften äquivalent. \begin{enumerate}
			\item $f$ ist stetig
			\item Für jede Menge $A \subset X$ ist $f(\overline{A}) \subseteq \overline{f(A)}$
			\item Für jede in $Y$ abgeschlossene Menge $F \subset Y$ ist $f^{-1}(Y)$ abgeschlossen in $X$
			\item zu jedem $x \in X$ und jeder Umgebung $V$ von $f(x)$ gibt es eine Umgebung $U$ von $x$, sodass $f(U) \subseteq V$.
		\end{enumerate}
	\end{theorem}
	\begin{definition}
		Eine bijektive Abbildung $f:X\to Y$ sodass $f$ und $f^{-1}$ stetig sind, heißt Homöomorphismus.
	\end{definition}
	\begin{definition}
		Ist $f$ eine stetige, injektive aber nicht surjektive Abbildung, so sei $\overline{Z} = f(X) \subset Y$. Die Einschränkung \[\tilde{f}: X \to \overline{Z}\] ist bijektiv. Ist $\tilde{f}$ ein Homöomorphismus, so nennen wir $f$ eine Einbettung von $X$ in $Y$.
	\end{definition}
	\begin{definition}
		Existiert zwischen $X$ und $Y$ ein Homöomorphismus, so heißen $X$ und $X$ homöomorph, geschrieben $X \cong Y$.
	\end{definition}
	\begin{theorem}
		Seien $X,Y,Z$ topologische Räume. Dann gilt \begin{enumerate}
			\item jede konstante Funktion $f:X\to Y$ ist stetig
			\item ist $A\subset X$, so ist die Inklusion $f: A \to X$ stetig
			\item sind $f:X\to Y$ stetig und $g:Y \to Z$ stetig, so ist $g\circ f$ stetig
			\item ist $f:X\to Y$ stetig und $A\subset X$, so ist die Restriktion $f:X \to A$ stetig bzgl. der Spurtopologie.
		\end{enumerate}
	\end{theorem}
	\begin{theorem}
		Seien $X,Y$ topologische Räume und es gelte $X = A\cup B$ mit $A,B$ abgeschlossen. Weiter seien die beiden Funktionen $f:A\to Y$ und $g:B \to Y$ stetig bzgl. der jeweiligen Spurtopologien und sei \[f(x) = g(x) \; \forall x \in A \cap B\] dann ist \[h(x) = \begin{cases}
			f(x), & x \in A\\
			g(x), & x \in B
		\end{cases}\] stetig auf $X$.
	\end{theorem}
	\begin{theorem}
		Es sei \[f:A \to X\times Y\] eine vektorwertige Funktion. $f$ ist stetig genau dann, wenn $f_1$ und $f_2$ stetig sind.
	\end{theorem}
	\section{Zusammenhängende topologische Räume}
	\begin{definition}
		Ein topologischer Raum $X$ heißt unzusammenhängend, wenn $\exists U,V$ offen, sodass $U\cap V = \varnothing$, $U \neq \varnothing, V \neq \varnothing$ und $X = U\cup V$. $X$ heißt zusammenhängend, wenn es keine solche Separation gibt. Das ist äquivalent dazu, dass $X$ und $\varnothing$ die einzigen abgeschloffenen Mengen sind.
	\end{definition}
	\begin{lemma}
		Sei $X$ ein topologischer Raum und sei $Y\subset X$. Eine Separation von $Y$ bzgl. der Spurtopologie ist ein Paar $(A,B)$ mit \[A\neq \varnothing, B \neq \varnothing, A\cup B = Y, \overline{A}\cap B = \varnothing, A \cap \overline{B} = \varnothing\]
	\end{lemma}
	\begin{lemma}
		Sei $X$ ein unzusammenhängender topologischer Raum mit Separation $(U,V)$. Ist $Y\subset X$ zusammenhängend, so ist $Y\subset U$ oder $Y\subset V$. 
	\end{lemma}
	\begin{theorem}
		Die Vereinigung zusammenhängender Räume ist zusammenhängend, wenn alle Räume mindestens einen Punkt gemeinsam haben.
	\end{theorem}
	\begin{theorem}
		Sind $X,Y$ zusammenhängende Räume, so ist $X\times Y$ zusammenhängend.
	\end{theorem}
	\begin{theorem}
		Sei $A\subset X$ ein zusammenhängender Teilraum. Dann ist $B$ mit $A\subset B \subset \overline{A}$ zusammenhängend.
	\end{theorem}
	\begin{theorem}
		Das stetige Bild eines zusammenhängenden Raumes ist zusammenhängend.
	\end{theorem}
	\begin{definition}
		Sei $X$ ein topologischer Raum. \begin{enumerate}
			\item Für $x,y \in X$ definieren wir einen Weg von $x$ nach $y$ als stetige Abbildung \[\gamma: [a,b] \to X\] mit \[\gamma(a) = x, \; \gamma(b) = y\]
			\item $X$ ist weg-zusammenhängend, wenn es für alle $x,y$ einen Weg $\gamma$ gibt, der in $X$ enthalten ist
		\end{enumerate}
	\end{definition}
	\begin{lemma}
		Weg-zusammenhängend $\Rightarrow$ zusammenhängend aber umgekehrt nicht.
	\end{lemma}
	\begin{definition}
		Für einen topologischen Raum $X$ bilden wir eine Äquivalenzrelation $\sim$ über \[x,y \in X: x \sim y \Leftrightarrow \exists Y \subset X: Y \text{ zusammenhängend mit } x,y \in Y\] 
	\end{definition}
	\begin{theorem}
		Zusammenhangskomponenten eines topologischen Raums $X$ sind zusammenhängende, paarweise disjunkte Teilmengen von $X$ deren Vereinigung $X$ ist. Das sind genau die Äquivalenzklassen von $\sim$. Jede zusammenhängende Teilmenge von $X$ ist ganz in einer der Zusammenhangskomponenten enthalten.
	\end{theorem}
	\section{Kompakte Topologische Räume}
	\begin{definition}
		Gegeben ein topologischer Raum $(X,\mathcal{T})$. Sei $\mathcal{A}\subset \mathcal{P}(X)$. $\mathcal{A}$ ist eine Überdeckung, wenn \[X = \bigcup_{A \in \mathcal{A}} A\] Gilt außerdem $\mathcal{A} \subset \mathcal{T}$, so spricht man von einer offenen Überdeckung. Wenn es zu jeder offenen Überdeckung $\mathcal{A}$ eine endliche Teilüberdeckung gibt, so ist $X$ kompakt.
	\end{definition}
	\begin{lemma}
		Sei $X$ ein topologischer Raum und $Y\subset X$. Dann ist $Y$ abgeschlossen bezüglich der Spurtopologie genau dann, wenn es $\mathcal{A} \subset \mathcal{T}$ gibt, sodass \[Y \subset \bigcup_{A \in \mathcal{A}} A\] und es von dieser Überdeckung eine endliche Teilüberdeckung gibt.
	\end{lemma}
	\begin{theorem}
		Abgeschlossene Teilräume eines kompakten Raums $X$ sind kompakt.
	\end{theorem}
	\begin{theorem}
		Jeder kompakte Teilraum eines Hausdorff-Raums ist abgeschlossen.
	\end{theorem}
	\begin{theorem}
		Das stetige Bild eines kompakten topologischen Raums ist wieder kompakt.
	\end{theorem}
	\begin{remark}
		Kompaktheit ist eine topologische Eigenschaft.
	\end{remark}
	\begin{theorem}
		Ist $f:X\to Y$ eine bijektive stetige Abbildung, $X$ kompakt und $Y$ Hausdorffsch, so ist $f$ bereits ein Homöomorphismus.
	\end{theorem}
	\begin{theorem}
		Das kartesische Produkt endlich vieler kompakter topologischer Räume ist wieder kompakt.
	\end{theorem}
	Es gibt auch die Begriffe der Häufungspunkt- und Folgenkompaktheit. Diese sind schwächer als die klassische Kompaktheit, aber in metrischen Räumen äquivalent.
	\begin{definition}
		\begin{enumerate}
			\item Ein topologischer Raum $X$ heißt lokal kompakt bei $x\in X$, wenn es einen kompakten Teilraum $C$ von $X$ gibt, der eine Umgebung von $X$ enthält.
			\item $X$ heißt lokal kompakt, wenn er bei allen Punkten $x\in X$ lokal kompakt ist.
		\end{enumerate}
	\end{definition}
	\subsection{Riemann-Sphäre}
	Die komplexe Ebene kann kompaktifiziert werden, indem die Ebene homöomorph auf eine Kugel mit fehlendem Punkt abgebildet wird. Der Kugelmittelpunkt sei bei $(0,0,\frac{1}{2})^T$ mit Radius $\frac{1}{2}$. Der Homoömorphismus bildet eine komplexe Zahl $z=x+iy$ ab auf $S(z)$. Die Koordinaten auf der Riemann-Sphäre werden als $\xi,\eta,\zeta$ bezeichnet. Die Formel für die Sphäre ist dann \[\xi^2+\eta^2+(\zeta-\frac{1}{2})^2 = \frac{1}{4}\]
	Die stereografische Abbildung $S(z)$ bildet nun jedes $z\in\mathbb{C}$ ab auf den Schnittpunkt der Verbindungsgerade von $z$ mit dem Nordpol $(0,0,1)^T$ der Kugel. Man kann sich davon überzeugen, dass $S$ ein Homöomorphismus ist.\\
	Die Kugel kann durch die Zunahme eines einzelnen Punktes (des Nordpols) kompaktifiziert werden. Da es sich um einen Homöomorphismus handelt, kann auch $\mathbb{C}$ kompaktifiziert werden (indem $\infty$ hinzugenommen wird).
	\begin{theorem}[1-Punkt Kopaktifizierung]
		Sei $X$ ein lokal kompakter Hausdorff-Raum. Dann gibt es einen Raum $Y$ mit folgenden Eigenschaften \begin{enumerate}
			\item $X$ ist ein Teilraum von $Y$
			\item $Y\setminus X$ besteht aus genau einem Element.
			\item $Y$ ist ein kompakter Raum
		\end{enumerate}
	\end{theorem}
\section{Homotope Wege}
	Die Idee ist, topologische Räume mit einer Gruppe zu identifizieren. Homöomorphe Räume haben dann isomorphe Gruppen. Auf diese Weise kann man die Überprüfung von Homöomorphie zurückführen auf die Überprüfung von Isomorphie.
	\begin{definition}
		Gegeben seien $X,Y$ topologische Räume und $f,f'$ zwei stetige Abbildungen von $X$ nach $Y$. Gibt es eine stetige Funktion $F:X\times [0,1] \to Y$ mit \[F(x,0) = f\] und \[F(x,1) = f'\] $\forall x \in X$, so heißen $f$ und $f'$ homotop. $F$ heißt Homotopie. Ist $f'$ zusätzlich konstant, so heißt $f$ nullhomotop. Man schreibt $f\simeq f'$.
	\end{definition}
	\begin{definition}
		Seien $f,f': [0,1] \to X$ Wege. Wir nennen sie Wege, wenn sie dieselben Anfangs- und Endpunkte ($x_0$ und $x_1$) haben und es eine Abbildung $F: [0,1]\times [0,1] \to X$ gibt mit \[F(s,0) = f, \; F(s,1) = f' \; \forall s \in [0,1]\] und \[F(0,t) = x_0, \; F(1,t) = x_1 \; \forall t \in [0,1]\]
		Ein solches $F$ heißt Weghomotopie. Man schreibt $f \simeq_p f'$.
	\end{definition}
	\begin{lemma}
		$\simeq$ und $\simeq_p$ sind Äquivalenzrelationen.
	\end{lemma}
	\begin{definition}
		Ist $f$ ein Weg, so ist $[f]$ die Äquivalenzklasse bezüglich $\simeq_p$.
	\end{definition}
	\begin{definition}
		Wir bezeichnen mit $f*g$ die Komposition zweier Wege $f$ und $g$. Dann ist $[f]*[g] = [f*g]$ eine Operation auf den Äquivalenzklassen.
	\end{definition}
	\begin{remark}
		Diese Operation erlaubt Assoziativität, die Existenz eines neutralen Elements und die Existenz eines inversen Element. Trotzdem ist die Menge der Homotopieklassen mit $*$ keine Gruppe, da $[f]*[g]$ nur definiert ist, wenn der Endpunkt von $f$ dem Anfangspunkt von $g$ entspricht. Daher liegt keine Gruppe vor sondern nur eine \textit{gruppoide Struktur}.
	\end{remark}
\end{document}